\documentclass{article}

\usepackage[spanish]{babel}
\usepackage{amsmath, amsthm, amssymb}
\usepackage{mathtools}
\usepackage{tikz}
\usepackage{enumitem} 

\usepackage{bbm}

\newtheorem{Teorema}{Teorema}
\newtheorem{Lema}{Lema}
\newtheorem{Proposicion}{Proposición}
\newtheorem{Corolario}{Corolario}

\theoremstyle{plain}
\newtheorem{Problema}{Problema}

\theoremstyle{definition}
\newtheorem{Definicion}{Definición}

\newcommand{\1}[1]{\mathbbm{1}\left( #1 \right)}

\newcommand{\T}{\mathbb{T}}
\newcommand{\C}{\mathbb{C}}
\newcommand{\R}{\mathbb{R}}
\newcommand{\Z}{\mathbb{Z}}
\newcommand{\N}{\mathbb{N}}
\newcommand{\F}{\mathcal{F}}
\newcommand{\K}{\mathbb{K}}

\DeclareMathOperator{\esssup}{esssup}

\newcommand{\abs}[1]{\lvert #1 \rvert}
\newcommand{\norm}[1]{\lVert #1 \rVert}
\newcommand{\normL}[2]{\lVert #2 \rVert_{L^#1}}

\begin{document}

\noindent\tikz{
\draw (0,0)--(\linewidth, 0) 
node[midway,above,fill=white]{\Large Espacios de Lebesgue};}
\newline

De aquí en adelante \((\Omega,\mathca{F},\mu)\) es un espacio de medida. Si se requieren
más se pondrán subíndices.

Recordamos que \(\mathcal{L}^{p}(\mu)\) es el conjunto de funciones medibles
\(p\)-integrables. La función \(\norm{\cdot}_p\) define una seminorma y tomando
\(\mathcal{N}(\mu)\) como las funciones que se anulan casi en todas partes podemos definir
el espacio
\begin{displaymath}
  L^p(\mu) \coloneqq \mathcal{L}^p(\mu) / \mathca{N}(\mu)
\end{displaymath}
que sí es un espacio normado con la norma \(\norm{\cdot}_{p}\). De hecho, es un espacio de Banach.

\begin{Teorema}[Riesz-Fischer]
  \(L^p(\mu)\) es un espacio de Banach. 
\end{Teorema}
\begin{proof}
  Vamos a probar que todas las series absolutamente sumables son sumables.

  \underline{Caso \(1\le p < \infty\):}
  Sea \((f_n)_{n\in\N}\) con \(f_n\in L^{p}(\mu)\) tal que
  \begin{equation}\label{eq:1}
    \sum_{n\in\N} \norm{f_n}_{p} \le M < \infty.
  \end{equation}
  Consideremos las sumas parciales puntuales \(G_n(x) = \sum_{k=1}^{n} \abs{f_k(x)}\). Nótese
  que son no negativas y crecientes. Además, \(G_n\) es una suma finita de funciones medibles,
  así que es medible. Por otro lado, \(G_n(x) \uparrow \sum_{n\ge 1} \abs{f_n(x)} \eqqcolon G(x)\)
  que está dominada por la serie en~\eqref{eq:1}. Así, por el Teorema de Convergencia Dominada
  (TCD) tenemos que
  \begin{equation}
    \lim_{n\to\infty} \int G_n(x) \, d\mu
    =
    \int G(x) \, d\mu.
  \end{equation}
  En particular,
  \begin{equation}
    \lim_{n\to\infty} \int G_n(x)^p \, d\mu
    =
    \int G(x)^p \, d\mu
    \eqqcolon I.
  \end{equation}
  Y por lo tanto \(I\le M^p\) pues basta tomar límite en la expresión:
  \begin{displaymath}
    \left( \int G_n(x)^p \right)^{1/p}
    =
    \norm{G_n}_p
    \le 
    \sum_{n\in\N} \norm{f_n}_{p} \le M < \infty.
  \end{displaymath}
  Esto nos dice que \(G^p \in L^1(\mu)\) y juntándolo con lo anterior concluimos que 
  \(0\le G^p < \infty\) \(\mu\)-ctp. De esta forma, el candidato a límite de la serie
  es
  \begin{equation}
    F(x) = 
    \begin{cases}
      \sum_{n\in\N} f_n(x) &, G(x) < \infty\\
      0 &, e.o.c
    \end{cases}
  \end{equation}
  Dado que \(\abs{F(x)} \le G(x)\), \(F\in L^{p}(\mu)\). Además, \(F\) es medible porque TODO.
  Queda ver la convergencia. Dado que \(\abs{F(x) - \sum_{k=1}^{n} f_k(x)}\) está dominado
  por \(G\) para cada \(x\), en particular \(\abs{F(x) - \sum_{k=1}^{n} f_k(x)}^p\)
  está dominado por \(G^p\). Luego, por TCD nos queda
  \begin{equation}
    \lim_{n\to\infty}
    \int \abs{F(x) - \sum_{k=1}^{n} f_k(x)}^p
    =
    0.
  \end{equation}
\end{proof}

\subsection*{El Teorema de Representación de Riesz}

Sea \(1\le p \le \infty\) y \(q\) su exponente conjuntado (\(1/q+1/p = 1\)). Definamos
\begin{equation}
\begin{aligned}
  \langle \cdot,\cdot \rangle\colon L^p(\mu)\times L^q(\mu) &\to \K\\
  f,g &\mapsto \int fg\, d\mu.
\end{aligned}
\end{equation}
Notar que el mapa está bien definido, en efecto, aplicando Hölder:
\begin{displaymath}
  \int \abs{fg} \, d\mu \le \norm{f}_p \norm{g}_q < \infty.
\end{displaymath}
Esto nos permite definir un funcional lineal en \(L^p(\mu)\) para cada elemento de 
\(L^q(\mu)\) dado por:
\begin{displaymath}
  l_{g}f \coloneqq \langle f,g \rangle.
\end{displaymath}
La linealidad viene por la linelidad de la integral y \(\norm{l_g}_{(L^p(\mu))^{\ast}} \le \norm{g}_{q}\). 
Más aún, el mapa \(\Phi\colon L^q(\mu) \to (L^p(\mu))^{\ast}\) dado por \(g\mapsto l_g\) es lineal,
acotado e inyectivo.

\begin{Teorema}[Representación de Riesz]
  Supongamos que \((\Omega,\mathcal{F},\mu)\) es \(\sigma\)-finito. Si \(1\le p < \infty\), entonces
  \(\Phi\) es un isomorfismo isométrico. 
\end{Teorema}

El resto de esta sección está dedicada a probar este teorema. Algunas consecuencias:
\begin{enumerate}
  \item \((\ell^p)^{\ast} \cong \ell^q\), esto es sale del teorema al poner el espacio de medida
  \((\N,\mathcal{P}(\N),\mu)\) donde \(\mu\) es la medida de contar. 
  \item 
\end{enumerate}

\begin{Teorema}[Radon-Nikodýn]
  Sean \(\mu,\nu\) medidas \(\sigma\)-finitas tal que \(\nu \ll \mu\). Entonces existe una única
  función \(h\) no negativa medible tal que \(\nu(E) = \int_{E} h\, d\mu\).  
\end{Teorema}

Recordar que como casi todo en medida, la unicidad es \(\mu\)-ctp al igual que lo es la no negatividad. 
Por otro lado, nótese que el teorema es válido para funciones a valores reales. No obstante, considerando
la parte real e imaginaria se puede extender a valores complejos. Por último, 
se usa la notación \(h = [\frac{d\nu}{d\mu}]\) y se dice que \(h\) es la derivada de Radon-Nikodýn. 

\begin{proof}
  La demostración se divide en dos pasos. En primer lugar lo probaremos para medidas finitas, y luego
  lo extendemos para medidas \(\sigma\)-finitas. El argumento de usar la Representación de Riesz
  para espacios de Hilbert se atribuye a Von Neumann.

  \underline{Paso I:} Supongamos que \(\mu\) y \(\nu\) son finitas. Consideremos la medida
  \(\lambda = \mu + \nu\). Notar que \(\lambda(E) = 0 \iff \mu(E) = 0\) pues \(\nu \ll \mu\).
  Consideremos el mapa 
  \begin{align*}
    l \colon L^{2}(\lambda) &\mapsto \R\\
    f &\mapsto \int f \, d\mu.
  \end{align*}
  El mapa es lineal por la linealidad de la integral y es acotado porque \(\mu\) es finita
  .i.e. \(\norm{l} \le \mu(\Omega)^{1/2}\). Así, por el Teorema de Representación de Riesz para 
  espacios de Hilbert, existe un único \(g\in L^{2}(\lambda)\) tal que
  \begin{displaymath}
    \int f \, d\mu = \int fg\, d\lambda
    \quad\forall f\in L^2(\lambda).
  \end{displaymath}
  Como las medidas son finitas, se sigue que
  \begin{equation}\label{eq:RN:proof1}
    \int f(1-g)\, d\mu = \int fg \, d\nu
    \quad\forall f\in L^2(\lambda).
  \end{equation}
  Notar que \(0 < g \le 1\) \(\lambda\)-ctp (que es lo mismo que \(\mu\)-ctp). En efecto,
  \begin{align*}
    \mu(\left\{ g \le 0 \right\})
    &\le
    \int \1{g \le 0} \, d\mu
    \\&\le
    \int \1{g \le 0} (1-g) \, d\mu
    \\&\overset{\eqref{eq:RN:proof1}}{=}
    \int \1{g \le 0} g d\nu
    \le 0.
  \end{align*}
  De manera similar,
  \begin{displaymath}
    0 >
    \int \1{1-g < 0} (1-g) \, d\mu
    =
    \int \1{1-g < 0} g \, d\nu
    \ge 0.
  \end{displaymath}
  Así que podemos tomar un representa que esté estrictamente entre \(0 < g \le 1\) y 
  podemos definir \(h = \frac{1-g}{g}\). Sea \(E\) un conjunto medible y definamos la sucesión
  de funciones \(f_n \coloneqq \1{E\cap\left\{ g\ge 1/n \right\}} \frac{1}{g}\). Como la medida
  es finita, \(f_n\in L^2(\lambda)\) y aplicando la Ecuación~\eqref{eq:RN:proof1} nos queda
  \begin{displaymath}
    \int f_n (1-g) \, d\mu = \int f_n g \, d\nu,
  \end{displaymath}
  es decir,
  \begin{displaymath}
    \int_{E\cap \left\{ g \ge 1/n \right\}} \frac{1-g}{g} \, d\mu
    =
    \int_{E\cap \left\{ g \ge 1/n \right\}} \, d\nu.
  \end{displaymath}
  Dado que \(E\cap \left\{ g\ge 1/n \right\} \uparrow E\), se sigue el resultado.
  
  \underline{Paso 2:} Supogamos ahora que \(\mu\) y \(\nu\) son \(\sigma\)-finitas. Sean 
  \(\Omega_n\) tal que \(\Omega_n \uparrow \Omega\) y cada uno tiene medida finita.
  Por lo anterior, para cada \(n\) existe una función \(h_n\) tal que
  \(\nu(E) = \int h_n \1{E} \, d\mu\) con \(E \in \mathcal{F}\cap \Omega_n \eqqcolon S_n\). 
  Además, \(\nu(E) = \int h_{n+1} \1{E} \, d\mu\) pues \(\Omega_n \subset \Omega_{n+1}\). Se sigue
  que \(h_{n+1}\vert_{\Omega_n} = h_n\) (igualdad ctp). Extendamos cada \(h_n\) por cero fuera
  de \(\Omega_n\). Nótese que \(h_n\) sigue siendo \(S_n\) medible. Definamos
  \(h = \lim_{n\to \infty} h_n\). Es claro que \(h_n \uparrow h\) y para todo \(E\) \(\mathcal{F}\)    
  medible se tiene que \(\nu(E) = \lim_{n\to\infty} \nu(\Omega_n\cap E)\). Usando TCM concluimos que
  \begin{displaymath}
    \nu(E) 
    = \lim_{n\to\infty} \int_{E\cap\Omega_n} h_n \, d\mu
    = \lim_{n\to\infty} \int_{E\cap\Omega} h \, d\mu
    = \int_{E} h \, d\mu.
  \end{displaymath}
\end{proof}

\begin{Definicion}[Operador Positivo]
  Decimos que \(t \in (L^p_{\R})^{\ast}\) es positivo si \(t(f) \ge 0\) para todo \(f\in L^{p}_{\R}\).  
\end{Definicion}

\begin{Teorema}[Descomposición de Funcionales]
  Sea \(t\in (L^{p}_{\R})^{\ast}\) con \(1\le p< \infty\), entonces 
  \(t = t_{+} - t_{-}\) con \(t_{\pm}\) un funcional positivo.  
\end{Teorema}
\begin{Demostracion}
\begin{enumerate}[label={Paso~\theenumi:}]
  \item Definamos \(t_{+}\) para funciones positivas \(f\in L^{p}_{\R}\) como
  \begin{displaymath}
    t_{+} \coloneqq \sup_{0 \le g \le f} t(g).
  \end{displaymath}
  Vamos a probar que \(t_{+}\) es lineal. Observemos que
  si \(0\le g \le f\) entonces \(t(g) \le t(f)\). Sean \(f_1,f_2 \in L^{p}_{\R}\). Sean 
  \(0 \le g_i \le f_i\). Luego, \(t(g_i) \le t(f_i)\) y por lo tanto
  \(\sup_{0\le g_i \le f_i} t(g_i) \le t(f_i)\). Sumando ambos términos y tomando
  supremo sobre los \(0 \le h \le f_1 + f_2\) nos queda:
  \begin{displaymath}
    t_{+}(f_1) + t_{+}(f_2) \le t_{+}(f_1 + f_2).
  \end{displaymath}
\end{enumerate}
  
\end{Demostracion}

\end{document}
