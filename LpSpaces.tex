\documentclass{article}

\usepackage[spanish]{babel}
\usepackage{amsmath, amsthm, amssymb}
\usepackage{mathtools}
\usepackage{tikz}

\newtheorem{Teorema}{Teorema}
\newtheorem{Lema}{Lema}
\newtheorem{Proposicion}{Proposición}
\newtheorem{Corolario}{Corolario}

\theoremstyle{plain}
\newtheorem{Problema}{Problema}

\theoremstyle{definition}
\newtheorem{Definicion}{Definición}

\newcommand{\T}{\mathbb{T}}
\newcommand{\C}{\mathbb{C}}
\newcommand{\R}{\mathbb{R}}
\newcommand{\Z}{\mathbb{Z}}
\newcommand{\N}{\mathbb{N}}
\newcommand{\F}{\mathcal{F}}

\DeclareMathOperator{\esssup}{esssup}

\newcommand{\abs}[1]{\lvert #1 \rvert}
\newcommand{\norm}[1]{\lVert #1 \rVert}
\newcommand{\normL}[2]{\lVert #2 \rVert_{L^#1}}

\begin{document}

\noindent\tikz{
\draw (0,0)--(\linewidth, 0) 
node[midway,above,fill=white]{\Large Espacios de Lebesgue};}
\newline

De aquí en adelante \((\Omega,\mathca{F},\mu)\) es un espacio de medida. Si se requieren
más se pondrán subíndices.

Recordamos que \(\mathcal{L}^{p}(\mu)\) es el conjunto de funciones medibles
\(p\)-integrables. La función \(\norm{\cdot}_p\) define una seminorma y tomando
\(\mathcal{N}(\mu)\) como las funciones que se anulan casi en todas partes podemos definir
el espacio
\begin{displaymath}
  L^p(\mu) \coloneqq \mathcal{L}^p(\mu) / \mathca{N}(\mu)
\end{displaymath}
que sí es un espacio normado con la norma \(\norm{\cdot}_{p}\). De hecho, es un espacio de Banach.

\begin{Teorema}[Riesz-Fischer]
  \(L^p(\mu)\) es un espacio de Banach. 
\end{Teorema}
\begin{proof}
  Vamos a probar que todas las series absolutamente sumables son sumables.

  \underline{Caso \(1\le p < \infty\):}
  Sea \((f_n)_{n\in\N}\) con \(f_n\in L^{p}(\mu)\) tal que
  \begin{equation}\label{eq:1}
    \sum_{n\in\N} \norm{f_n}_{p} \le M < \infty.
  \end{equation}
  Consideremos las sumas parciales puntuales \(G_n(x) = \sum_{k=1}^{n} \abs{f_k(x)}\). Nótese
  que son no negativas y crecientes. Además, \(G_n\) es una suma finita de funciones medibles,
  así que es medible. Por otro lado, \(G_n(x) \uparrow \sum_{n\ge 1} \abs{f_n(x)} \eqqcolon G(x)\)
  que está dominada por la serie en~\eqref{eq:1}. Así, por el Teorema de Convergencia Dominada
  (TCD) tenemos que
  \begin{equation}
    \lim_{n\to\infty} \int G_n(x) \, d\mu
    =
    \int G(x) \, d\mu.
  \end{equation}
  En particular,
  \begin{equation}
    \lim_{n\to\infty} \int G_n(x)^p \, d\mu
    =
    \int G(x)^p \, d\mu
    \eqqcolon I.
  \end{equation}
  Y por lo tanto \(I\le M^p\) pues basta tomar límite en la expresión:
  \begin{displaymath}
    \left( \int G_n(x)^p \right)^{1/p}
    =
    \norm{G_n}_p
    \le 
    \sum_{n\in\N} \norm{f_n}_{p} \le M < \infty.
  \end{displaymath}
  Esto nos dice que \(G^p \in L^1(\mu)\) y juntándolo con lo anterior concluimos que 
  \(0\le G^p < \infty\) \(\mu\)-ctp. De esta forma, el candidato a límite de la serie
  es
  \begin{equation}
    F(x) = 
    \begin{cases}
      \sum_{n\in\N} f_n(x) &, G(x) < \infty\\
      0 &, e.o.c
    \end{cases}
  \end{equation}
  Dado que \(\abs{F(x)} \le G(x)\), \(F\in L^{p}(\mu)\). Además, \(F\) es medible porque TODO.
  Queda ver la convergencia. Dado que \(\abs{F(x) - \sum_{k=1}^{n} f_k(x)}\) está dominado
  por \(G\) para cada \(x\), en particular \(\abs{F(x) - \sum_{k=1}^{n} f_k(x)}^p\)
  está dominado por \(G^p\). Luego, por TCD nos queda
  \begin{equation}
    \lim_{n\to\infty}
    \int \abs{F(x) - \sum_{k=1}^{n} f_k(x)}^p
    =
    0.
  \end{equation}
\end{proof}

\end{document}
