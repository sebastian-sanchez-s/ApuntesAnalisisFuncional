\documentclass[11pt]{article}

% Page set up
\usepackage[margin=3cm]{geometry}

% Document font and symbols
%\usepackage{mathptmx} % Times New Roman
\usepackage{amsmath,amsfonts,amsthm,amssymb}
\usepackage{esint}

% Language formatting
\usepackage[utf8]{inputenc}
\usepackage[T1]{fontenc}
\usepackage[spanish, es-noshorthands]{babel}

% Graph and Drawings
\usepackage{tikz,tikz-cd,float}
\usepackage{wrapfig}
\usetikzlibrary{babel}
\usepackage{pgfplots}
\usepackage[framemethod=default]{mdframed}
\usepackage{framed}

% Tools
\usepackage{multicol}
\usepackage{array}
\usepackage{hyperref}
\usepackage{xcolor}
\usepackage{etoolbox,mathtools}
\usepackage[shortlabels]{enumitem}

% Mdframed template
\mdfsetup{innertopmargin=5pt,%
	frametitlealignment=\raggedright,%
	frametitlefont=\texttt,%
	linewidth=2pt,%
	topline=false,rightline=false, bottomline=false,%
	frametitleaboveskip=\dimexpr-1\ht\strutbox\relax,}

% Problem Env
\newcounter{ProblemCounter}
\newenvironment{Problema}[1][]{%
	\vspace{1em}
	\refstepcounter{ProblemCounter}
    \noindent\texttt{PROBLEMA}~\theProblemCounter~
	}{
	\ \newline
	}

% Solution Env
\newenvironment{Solucion}[1][]
{%
    \newline
	\noindent{\ttfamily SOLUCIÓN}~
}%
{%
	%\hfill\(\blacksquare\)
}

% Resize abs and norm
\DeclarePairedDelimiter{\abs}{\lvert}{\rvert}
\DeclarePairedDelimiter{\norm}{\|}{\|}
\makeatletter
\let\oldabs\abs
\def\abs{\@ifstar{\oldabs}{\oldabs*}}
\let\oldnorm\norm
\def\norm{\@ifstar{\oldnorm}{\oldnorm*}}
\makeatother

% Shortcuts
\newcommand{\K}{\mathbb{K}}
\newcommand{\R}{\mathbb{R}}
\newcommand{\N}{\mathbb{N}}

% Header
\newcommand{\header}[2]
{
\begin{minipage}{.15\textwidth}
	\includegraphics[width=\textwidth,height=\textheight,keepaspectratio]{LogoUC}
\end{minipage}
\hspace{1em}
\begin{minipage}{.75\textwidth}
	\vspace{2em}
	{\scshape
	Pontificia Universidad Católica de Chile\\
	Facultad de Matemáticas\\
	Docente: Nikola Kamburov\\
	Ayudante: Matías Díaz
	}
\end{minipage}

\medskip

\begin{center}
	{\textbf{MAT2555 {-} Análisis Funcional}} \\[1ex]
	{{\large Tarea #1 {-} Omar Neyra, Sebastián Sánchez}}
\end{center}

\medskip
}

\begin{document}

\header{1}{}

% Problema 1
\begin{Problema}
  Sea \(\Omega \subset \R^n\) abierto y \(\alpha \in (0,1]\). Para \(f\colon \Omega\to \K\),
  define su \textit{norma de Hölder}
  \begin{displaymath}
    \norm{f}_{C^{\alpha}}
    \coloneqq
    \sup_{x\in\Omega} \abs{f(x)}
    +
    \sup_{x\ne y \in \Omega}
    \frac{\abs{f(x) - f(y)}}{\abs{x-y}^{\alpha}}.
  \end{displaymath}
  El espacio \(C^{\alpha}(\Omega) \coloneqq \lbrace f\colon \Omega \to \K \mid 
  \norm{f}_{C^{\alpha}} < \infty \rbrace\)
  se llama el \textit{espacio de Hölder} de funciones en \(\Omega\), de exponente
  \(\alpha\). Demuestre que \((C^{\alpha}(\Omega), \norm{\cdot}_{C^{\alpha}})\) es
  un espacio de Banach.
\end{Problema}
\begin{Solucion}
Sea \((f_n)_{n\in\N}\) usa sucesión de Cauchy en \(X = C^{\alpha}(\Omega)\).
Debemos probar que su límite está en \(X\).

\underline{Candidato a límite}: En primer lugar, notemos que fijado
\(x\in\Omega\) tenemos que la sucesión \((f_n(x))_{n\in\N}\) es Cauchy
en \(\K\). En efecto,
\begin{displaymath}
    \abs{f_n(x) - f_m(x)}
    \le
    \sup_{x\in\Omega} \abs{f_n(x) - f_m(x)}
    \le
    \norm{f_n - f_m}_{X}
    \to 0
\end{displaymath}
para \(m,n\) suficientemente grandes. Así, el candidato a límite es:
\begin{displaymath}
    f(x) = \lim_{n\to\infty} f_n(x).
\end{displaymath}

\underline{Pertenencia del límite:} Usando la definición de \(f\) y la continuidad de \(\abs{\cdot}\)
tenemos:
\begin{align*}
    \norm{f}_{X}
    &=
    \sup_{x\in\Omega} \abs{f(x)}
    +
    \sup_{x\ne y\in \Omega}
    \frac
        { \abs{f(x) - f(y)} }
        { \abs{x - y} }
    \\&=
    \sup_{x\in\Omega} \lim_{n\to\infty} \abs{f_n(x)}
    +
    \sup_{x\ne y\in \Omega}
    \lim_{n\to\infty}
    \frac
        { \abs{f_n(x) - f_n(y)} }
        { \abs{x - y} }
\end{align*}
Usando propiedades de supremo y límites nos queda:
\begin{displaymath}
    \norm{f}_{X}
    \le
    \liminf_{n\to\infty} 
    \left(
    \sup_{x\in\Omega} \abs{f_n(x)}
    +
    \sup_{x\ne y\in \Omega}
    \frac
        { \abs{f_n(x) - f_n(y)} }
        { \abs{x - y}^{\alpha} }
    \right)
\end{displaymath}

\underline{Covergencia:} Finalmente, queda probar que \(f_n \to f\)
en norma \(\norm{\cdot}_{\infty}\). En efecto, usando el cálculo de la 
parte anterior tenemos:
\begin{displaymath}
    \norm{f - f_n}_{X}
    \le
    \liminf_{k\to\infty}
    \left(
    \sup_{x\in\Omega} \abs{f_k(x) - f_n(x)}
    +
    \sup_{x\ne y\in \Omega}
    \frac
        { \abs{f_k(x) - f_n(x) + f_n(y) - f_k(y)} }
        { \abs{x - y}^{\alpha} }
    \right)
\end{displaymath}
Aplicando desigualdad triangular en el último sumando y tomando
un \(n\) suficientemente grande tenemos que cada sumando se va a \(0\).
Concluyéndose lo buscado.
\end{Solucion}

% Problema 2
\begin{Problema}
  Considere el espacio 
  \begin{displaymath}
    c_0(\K) \coloneqq
    \lbrace 
    (a_k)_{k\in\N} \text{ en } \ell^{\infty}(\K)
    \colon
    \lim_{k\to\infty} a_k = 0
    \rbrace.
  \end{displaymath}
  \begin{enumerate}[(a)]
    \item Demuestre que \(c_0(\K)\) es cerrado en \(\ell^{\infty}(\K)\). 
    Concluya que \(c_0(\K)\) es un espacio de Banach. 

    \item Sea \(V\) el espacio cociente \(V = \ell^{\infty}/c_0\) con la norma
    inducida
    \begin{displaymath}
      \norm{[v]} = \inf_{w\in c_0} \norm{v + w}_{\ell^{\infty}}.
    \end{displaymath}
    Demuestre que \(\norm{[v]} = \limsup_{k\to\infty} \abs{a_k}\). 
  \end{enumerate}
\end{Problema}
\begin{Solucion}
  \framebox{(a)} Sea \(a = (a_k)_{k}\) en \(\ell^{\infty}\) tal que 
  \((a^{m})_{m} \to a\). Debemos probar que \(a\in c_0\). Observemos 
  que
  \begin{displaymath}
    \abs{a_k}
    = 
    \abs{a_k - a^{m}_{k}}
    + 
    \abs{a^m_k}.
  \end{displaymath}
  El primer sumando satisface
  \begin{displaymath}
    \abs{a_k - a^{m}_{k}}
    \le
    \sup_{k} \abs{a_k - a^{m}_{k}}
    = \norm{a - a^{m}}_{\ell^{\infty}}
    \xrightarrow{m\to\infty} 0
  \end{displaymath}
  pues \(a^{m} \to a\). Por otro lado,
  \(
    \abs{a^m_k} \to 0
  \) 
  pues \(a^m \in c_0\).
  Con ambas convergencias se sigue el resultado.
  
  \framebox{(b)}
  Expandamos la definición de la norma cociente:
  \begin{displaymath}
    \norm{[v]}
    = \inf_{w\in c_0} \norm{v + w}_{\ell^{\infty}}
    = \inf_{w\in c_0}\, \sup_{k\in\N}\, \abs{v_k + w_k}
  \end{displaymath}
  En particular, para \(w^{n} = (-v_1, -v_2, \dots, -v_n, 0, 0, \dots, 0)\in c_0\)
  se tiene que
  \begin{displaymath}
    \norm{[v]}
    \le
    \sup_{k>n}\, \abs{v_k}
    \Rightarrow
    \norm{[v]}
    \le
    \inf_{n} \sup_{k>n}\, \abs{v_k}
    =
    \limsup_{k\to\infty} \abs{v_k}.
  \end{displaymath}

  La otra desigualdad es directa:
  \begin{displaymath}
    \inf_{n} \sup_{k>n}\, \abs{v_k}
    \le
    \inf_{n} \sup_{k} \, \abs{v_k + w_k}
    \le
    \norm{[v]}.
  \end{displaymath}
  

\end{Solucion}

\begin{Problema}
  De un ejemplo de una sucesión \(a = (a_k)_{k\in\N}, a_k\in \C\)
  que,
  \begin{enumerate}[(a)]
      \item
      converge en \(\ell^{\infty}\), pero no en \(\ell^{1}\);
      \item
      converge en \(\ell^{2}\), pero no en \(\ell^{1}\);
      \item
      converge en \(c_0\), pero no en \(\ell^{2}\).
  \end{enumerate}
\end{Problema}
\begin{Solucion}
  \framebox{(a)} Definir \(a = (a_k)_k = (e^{ik})_k\). Luego, 
  \(\norm{a}_{\ell^{\infty}} = 1\) y \(\norm{a}_{\ell^{1}} \to \infty\).

  \framebox{(b)} Considerar \(a = (a_k)_k = (1/k)_k\). Luego,
  \(\norm{a}_{\ell^{2}} < \infty\) y \(\norm{a}_{\ell^{1}} \to \infty\). 

  \framebox{(c)} Considerar \(a = (a_k)_k = (1/\sqrt{k})_{k}\). Luego,
  \(\norm{a}_{\ell^{2}} \to \infty\) y \(\norm{a}_{\ell^{\infty}} \le 1\)
  y \(a_k \to 0\). 
\end{Solucion}

% Problema 4
\begin{Problema}
  Sea \((V,\norm{\cdot})\) un espacio de Banach.
  \begin{enumerate}
    \item Pruebe que si 
      \(
        F_k 
        \coloneqq \overline{B}_{r_k}(v_k) 
        \coloneqq \left\{ v\in V \colon \norm{v-v_k} \le r_k \right\}
      \)
      , \(k\in\N\), son bolas cerradas, tales que
      \(F_{k} \supseteq F_{k+1} \forall k\in \N\),
      entonces \(\bigcap_{k} F_k \ne \varnothing\). 
      
    \item ¿Es posible tener conjuntos cerrados \(F_k\), tales que 
      \(F_k \supseteq F_{k+1}\) y \(\bigcap_{k} F_k = \varnothing\)?  
  \end{enumerate}
\end{Problema}
\begin{Solucion}
\begin{enumerate}[(a)]
  \item
  En primer lugar, notemos que si los radios se estancan (es decir,
  \(r_k = r_{j}\) para todo \(j\ge k\) y \(k\) fijo) entonces necesariamente
  \(v_k = v_{j}\) para todo \(k\ge k\) y por lo tanto la intersección
  de todos es no vacía (\(v_k\) está en todos). 

  Por otro lado, si los radios no se estancan, necesariamente deben ir
  decreciendo, es decir, \(r_k \downarrow c\).
  Mostraremos que la sucesión de los centros \((v_k)_{k}\) es Cauchy.

  En efecto, sea \(\epsilon > 0\). Luego, existe \(k\) tal que 
  \(r_k - c < 2^{-k} \epsilon\). Sea \(j > k\), luego,
  \begin{displaymath}
    \norm{v_k - v_{j}} 
    \le
    \sum_{i=k}^{j-1} \norm{v_i - v_{i+1}}
    \le
    \sum_{i=k}^{j-1} \epsilon 2^{-i}
    =
    \frac{1}{2^{k-1}} - \frac{1}{2^{j-1}}
    \to 0
  \end{displaymath}
  Así, la sucesión \((v_k)_k\) es Cauchy y como \(V\) es Banach,
  el límite existe. Se sigue que \(v = \lim_{k\to\infty} v_k\) está en
  la intersección.

  \item
  Basta considerar \(F_k = \varnothing\) para todo \(k\).
\end{enumerate}
\end{Solucion}

% Problema 5
\begin{Problema}
  Sea \(V\) un espacio normado y sea \(l\) un funcional lineal acotado
  no cero en \(V\). Considere el espacio nulo de \(l\):
  \begin{displaymath}
    L = \left\{ v \in V \colon l(v) = 0 \right\}.
  \end{displaymath}

  \begin{enumerate}[(a)]
    \item
    Demuestre que
    \begin{displaymath}
      \abs{l(v)} = \norm{l} d(v,L)
      \text{ para todo} v\in V,
    \end{displaymath}
    donde \(d(v,L) \coloneqq \inf_{w\in L} \norm{v-w}\) es la distancia 
    entre \(v\) y \(L\).

    \item
    Pruebe que si existen \(v\in V\setminus L\), \(w\in L\),
    tal que \(\norm{v-w} = d(v,L)\), entonces \(l\) alcanza su
    norma en la esfera unitaria \(\left\{ x\in V\colon \norm{x} = 1 \right\}.\) 
  \end{enumerate}
\end{Problema}
\begin{Solucion}
\begin{enumerate}[(a)]
  \item 
  Para una desigualdad, notemos que:
  \begin{displaymath}
    \abs{l(v)} 
    = \abs{l(v - w + w)}
    = \abs{l(v-w)}
    \le \norm{l} \norm{v-w},
  \end{displaymath}
  dado que la desigualdad vale para todo \(w\), basta tomar el ínfimo
  para que se siga el resultado. Para la otra dirección, supongamos que
  \(\abs{l(v)} > \norm{l} d(v,L)\). Como \(0 \in L\) (pues \(l\) es un
  funcional lineal), se sigue que
  \begin{displaymath}
    \abs{l(v)} > \norm{l} d(v,L) \ge \norm{l} \norm{v},
  \end{displaymath}
  contradiciendo que \(l\) sea un funcional lineal acotado. 
  
  \item
  Sean \(v\) y \(w\) como en el enunciado. Definamos
  \(u = (v-w)/\norm{v-w} = (v-w)/d(v,L)\). Por la parte
  anterior se tiene que
  \begin{displaymath}
    \abs{l(u)} 
    = \norm{l} d(u,L)
    = \norm{l},
  \end{displaymath}
  pues
  \begin{displaymath}
    d(u,L)
    = \inf_{w'\in L} \norm{u - w'}
    = \frac{1}{d(v,L)} \inf_{w'\in L} \norm{v - w'}
    = \frac{d(v,L)}{d(v,L)} = 1.
  \end{displaymath}
\end{enumerate}
\end{Solucion}

% Problema 6
\begin{Problema}
  Sean \(V,W\) espacios de Banach, \(T \in B(V,W)\), \(\alpha \in (0,1)\) y 
  \(\beta > 0\). Supongamos que para todo \(w\in W\), existe \(v\in V\) tal
  que \(\norm{v} \le \beta \norm{w}\) y
  \begin{displaymath}
    \norm{Tv - w} \le \alpha \norm{w}.
  \end{displaymath}
  Pruebe que para todo \(w\in W\) dado, la ecuación \(Tv = w\) tiene una
  solución \(v\in V\) con \(\norm{v} \le \beta/(1-\alpha)\,\norm{w}\).
\end{Problema}
\begin{Solucion}
  Sea \(v_0\) el vector asociado con \(w\) según el enunciado. Luego,
  \begin{displaymath}
    \norm{v_0} \le \beta \norm{w}
    \text{ y }
    \norm{Tv_0 - w} \le \alpha \norm{w}.
  \end{displaymath}
  Consideremos \(v_1\) el vector asociado a \(w_1 = Tv_0-w\). Luego,
  \begin{displaymath}
    \norm{v_1} \le \beta \norm{Tv_0-w}
    \text{ y }
    \norm{Tv_1 - (Tv_0 - w)}
    \le \alpha \norm{Tv_0-w}.
  \end{displaymath}
  Descomponiendo todo para dejarlo en términos de \(w\) tenemos que
  \begin{displaymath}
    \norm{v_1} \le \beta \alpha \norm{w}
    \text{ y }
    \norm{Tv_1 - w_1} \le \alpha^2 \norm{w}
  \end{displaymath}
  Ahora consideremos \(v_2\) el vector asociado a \(w_2 = Tv_1 - w_1\). Se sigue
  que
  \begin{displaymath}
    \norm{v_2} \le \beta \alpha^2 \norm{w}
    \text{ y }
    \norm{Tv_2 - w_2} \le \alpha^{3} \norm{w}.
  \end{displaymath}
  Siguiendo el procedimiento tendremos 
  \begin{displaymath}
    \norm{v_n} \le \beta \alpha^{n} \norm{w}
    \text{ y }
    \norm{Tv_n - w_n} \le \alpha^{n+1} \norm{w}.
  \end{displaymath}
  Definamos \(v = \sum_{n\ge 0} v_n\). Afirmamos que \(v\) existe y que \(Tv = w\).
  Para lo primero, basta ver que la serie es absolutamente convergente porque
  estamos en un espacio de Banach:
  \begin{displaymath}
    \norm{v}
    \le \sum_{n \ge 0} \norm{v_n}
    \le \beta \norm{w} \sum_{n \ge 0} \alpha^{n}
    \le \beta \norm{w} \frac{1}{1-\alpha}
    < \infty,
  \end{displaymath}
  pues \(\alpha \in (0,1)\). Para lo segundo, debemos probar que la imágenes
  de las sumas parciales convergen a \(w\).
  \begin{alignat*}{2}
    \norm{w - \sum_{n=0}^{N} T v_n }
    &=
    \norm{(w-Tv_0) - \sum_{n=1}^{N} Tv_n}
    \\&=
    \norm{w_1 - \sum_{n=1}^{N} Tv_n}
    \\&=
    \norm{(w_1 - Tv_1) - \sum_{n=2}^{N} Tv_n}
    \\&=
    \norm{w_2 - \sum_{n=2}^{N} Tv_n}
    \\&\vdots
    \\&=
    \norm{w_N - Tv_N}
    \le
    \alpha^{N+1} \norm{w}.
  \end{alignat*}
  Tomando \(N\to\infty\) tenemos que \(Tv = w\). 
\end{Solucion}

% Problema 7
\begin{Problema}
  Sea \(X\) un espacio métrico completo y sea \((f_n)_{n}\) una sucesión de funciones
  continuas en \(X\) con valores en \(\K\), tal que \(f(x) = \lim_{n\to\infty} f_n(x)\)
  existe para todo \(x\in X\). En esta pregunta, demostrará que el conjunto donde la
  función límite \(f\) es continua, es denso en \(X\).
  \begin{enumerate}[(a)]
    \item Pruebe que para todo \(\epsilon > 0\) y toda bola cerrada \(\overline{B} \subset X\),
    existe una bola abierta \(\tilde{B} \subset B\) y un \(N\in \N\), tales que
    \begin{displaymath}
      \abs{f_N(x) - f(x)} \le \epsilon
      \qquad
      \text{para todo } x \in \tilde{B}.
    \end{displaymath}
    (\textit{Sugerencia:} Considere \(E_l \coloneqq \lbrace x\in \overline{B} \colon 
    \sup_{j,k\ge l} \abs{f_j(x) - f_k(x)} \le \epsilon \rbrace\) y note que
    \(\overline{B} = \bigcup_{l} E_{l}\)).

    \item
    Defina la oscilación de \(f\) en \(x\), 
    \begin{displaymath}
      osc(f)(x) = \lim_{r\downarrow 0} \sup_{y,z\in B_r(x)} \abs{f(y) - f(z)},
    \end{displaymath}
    y note que \(f\) es continua en \(x\) si y solo si \(osc(f)(x) = 0\). Demuestre que
    \begin{displaymath}
      F_t \coloneqq \lbrace x\in X\colon osc(f)(x) \ge t \rbrace, t>0,
    \end{displaymath}
    es cerrado y denso en ninguna parte. 
    (\textit{Sugerencia}: \(\lbrace x\in X \colon osc(f)(x) < t \rbrace\) es abierto
    para cualquier función \(f\)) 

    \item
    Concluya que el conjunto de discontinuidades de \(f\) es de la primera categoría en
    \(X\), i.e. el conjunto donde \(f\) es continua es genérico en \(X\).
  \end{enumerate}
\end{Problema}
\begin{Solucion}
\begin{enumerate}[(a)]
  \item
  Definamos \(E_l\) como en la sugerencia. En primer lugar, notemos que son cerrados.
  En efecto, sea \(x\) un punto límite de \(E_l\), entonces existe una sucesión \((x_n)_n\)
  en \(E_l\) tal que \(x_n \to x\). Para ver que \(x\in E_l\) notemos que descomponiendo
  el límite y usando la continuidad de \(f_i\) se tiene que
  \begin{displaymath}
    \sup_{j,k\le l} \abs{f_j(x) - f_k(x)}
    =
    \sup_{j,k\le l} \lim_{n\to\infty} \abs{f_j(x_n) - f_k(x_n)}
    \le \epsilon.
  \end{displaymath}
  Así, llegamos que la bola cerrada es unión de conjuntos cerrados y por Baire
  (la bola es completa al ser cerrada) se sigue existe \(N\) tal que \(E_N\) tiene interior
  no vacío. Sea \(\tilde{B}\) una bola abierta en \(E_N\), entonces se sigue que
  \begin{displaymath}
    \abs{f_N(x) - f_j(x)} \le \epsilon 
    \qquad\forall x\in \tilde{B}.
  \end{displaymath}
  Tomando \(j \to \infty\) se concluye lo pedido.

  \item

  
\end{enumerate}
\end{Solucion}

\end{document}

