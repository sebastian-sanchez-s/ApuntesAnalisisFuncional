\documentclass[11pt]{article}

% Page set up
\usepackage[margin=3cm]{geometry}

% Document font and symbols
\usepackage{mathptmx} % Times New Roman
\usepackage{amsmath,amsfonts,amsthm,amssymb}
\usepackage{esint}

% Language formatting
\usepackage[utf8]{inputenc}
\usepackage[T1]{fontenc}
\usepackage[spanish, es-noshorthands]{babel}

% Graph and Drawings
\usepackage{tikz,tikz-cd,float}
\usepackage{wrapfig}
\usetikzlibrary{babel}
\usepackage{pgfplots}
\usepackage[framemethod=default]{mdframed}
\usepackage{framed}

% Tools
\usepackage{multicol}
\usepackage{array}
\usepackage{hyperref}
\usepackage{xcolor}
\usepackage{etoolbox,mathtools}
\usepackage[shortlabels]{enumitem}

% Mdframed template
\mdfsetup{innertopmargin=5pt,%
	frametitlealignment=\raggedright,%
	frametitlefont=\texttt,%
	linewidth=2pt,%
	topline=false,rightline=false, bottomline=false,%
	frametitleaboveskip=\dimexpr-1\ht\strutbox\relax,}

% Problem Env
\newcounter{ProblemCounter}
\newenvironment{Problema}[1][]{%
	\vspace{1em}
	\refstepcounter{ProblemCounter}
  \noindent\tikz{\draw (0,0)--(\linewidth, 0) node[midway,fill=white]{{{\texttt{PROBLEMA}}~\theProblemCounter}};}
	}{
    \newline
    \noindent\tikz{\draw (0,0) -- (\linewidth, 0);}
	}

% Solution Env
\newenvironment{Solucion}[1][]
{%
  \newline
	\noindent{\ttfamily SOLUCIÓN}~
}%
{%
	%\hfill\(\blacksquare\)
}

% Resize abs and norm
\DeclarePairedDelimiter{\abs}{\lvert}{\rvert}
\DeclarePairedDelimiter{\norm}{\|}{\|}
\makeatletter
\let\oldabs\abs
\def\abs{\@ifstar{\oldabs}{\oldabs*}}
\let\oldnorm\norm
\def\norm{\@ifstar{\oldnorm}{\oldnorm*}}
\makeatother

% Shortcuts
\newcommand{\K}{\mathbb{K}}
\newcommand{\C}{\mathbb{C}}
\newcommand{\Q}{\mathbb{Q}}
\newcommand{\R}{\mathbb{R}}
\newcommand{\N}{\mathbb{N}}
\newcommand{\Z}{\mathbb{Z}}
\newcommand{\T}{\mathbb{T}}

\DeclareMathOperator{\esssup}{esssup}

% Header
\newcommand{\header}[1]
{
\begin{minipage}{.15\textwidth}
	\includegraphics[width=\textwidth,height=\textheight,keepaspectratio]{LogoUC}
\end{minipage}
\hspace{1em}
\begin{minipage}{.75\textwidth}
	\vspace{2em}
	{\scshape
	Pontificia Universidad Católica de Chile\\
	Facultad de Matemáticas\\
	Docente: Nikola Kamburov\\
	Ayudante: Matías Díaz
	}
\end{minipage}

\medskip

\begin{center}
	{\textbf{MAT2555 {-} Análisis Funcional}} \\[1ex]
	{{\large Tarea #1 {-} Omar Neyra, Sebastián Sánchez}}
\end{center}

\medskip
}

\begin{document}

\header{3}

% Problema 1
\begin{Problema}
  Sea \((\Omega, M, \mu)\) un espacio de medida y suponga que \(f\in L^{p_0}(\mu)
  \cap L^{\infty}(\mu)\) para algún \(p_0 \in [1,\infty)\). Pruebe que
  \(f\in L^p\) para todo \(p\ge p_0\) y que 
  \begin{displaymath}
    \norm{f}_{\infty} = \lim_{p\to\infty} \norm{f}_{p}.
  \end{displaymath}
\end{Problema}
\begin{Solucion}
  El caso \(p = p_0\) es directo, así que supongamos que la desigualdad es estricta y 
  denotemos \(p' \coloneqq p - p_0 > 0\). Notando que \(\abs{f(x)} \le \norm{f}_{\infty}\) 
  (c.t.p) tenemos que
  \begin{equation}
    \int \abs{f}^{p} 
    =
    \int \abs{f}^{p_0} \abs{f}^{p'}
    \le
    \int \abs{f}^{p_0} \norm{f}_{\infty}^{p'}
    \le
    \norm{f}_{\infty}^{p'} \norm{f}_{p_0}^{p_0}
  \end{equation}
  Todas las cantidades son positivas, así que tomando raíz obtenemos que
  \begin{equation}
    \norm{f}_{p} \le \norm{f}_{\infty}^{p'/p} \norm{f}_{p_0}^{p_0/p} < \infty,
  \end{equation}
  pues la norma uniforme y \(p_0\) están acotadas. Tomando límite se ve directamente que
  \begin{equation}
    \lim_{p\to\infty} \norm{f}_{p} \le \norm{f}_{\infty}
  \end{equation}
  pues \(p'/p = 1 - p_0/p \to 1\) y \(p_0/p \to 0\) cuando \(p\to \infty\).   

  Para la otra dirección, consideremos \(\epsilon > 0\). Luego,
  \begin{align*}
    \norm{f}_{p} 
    &= \left( \int \abs{f}^p \right)^{1/p}
    \\&\ge
    \left( \int_{\left\{ \abs{f} + \epsilon > \norm{f}_{\infty} \right\} } \abs{f}^{p} \right)^{1/p}
    \\&\ge
    \left( \int_{\left\{ \abs{f} + \epsilon > \norm{f}_{\infty} \right\} } (\norm{f}_{\infty} - \epsilon)^{p} \right)^{1/p}
    \\&\ge
    (\norm{f}_{\infty}-\epsilon) \mu( \left\{ \abs{f} + \epsilon > \norm{f}_{\infty} \right\} )^{1/p}.
  \end{align*}
  Notar que \(\mu(\left\{ \abs{f} + \epsilon > \norm{f}_{\infty} \right\}) < \infty\) pues \(f\in L^p\).
  Tomando límite tenemos que \(\lim_{p\to\infty} \norm{f}_p \ge \norm{f}_{\infty} - \epsilon\).
  Como \(\epsilon\) es arbitrario, se concluye el resultado. 
\end{Solucion}

% Problema 2
\begin{Problema}
  Para todo \(a\in\R\) construya una función \(f_a \in L^{\infty}(\R)\) con
  \(\norm{f_a - f_b}_{L^{\infty}(\R)} \ge 1\) cuando \(a\ne b\). Demuestre que esto implica
  que \(L^{\infty}(\R)\) no es separable. 
\end{Problema}
\begin{Solucion}
  Basta considerar \(f_a(x) = \chi_{[a,\infty]}(x)\). Para lo otro. Supongamos que
  es separable. Sea \(\left\{ x_n \right\}_{n}\) un conjunto denso y numerable.
  Fijemos \(1/4 > \epsilon > 0\). Luego, existe \(N\) tal que al menos para dos reales \(a\) y 
  \(b\) se tiene que \(f_a, f_b \in B(x_N, \epsilon)\) (Un palomar, hay numerables
  bolas y tenemos no-numerables elementos que disponer). De esta forma:
  \begin{displaymath}
    1 = \norm{f_a - f_b} \le \norm{f_a - x_N} + \norm{f_b - x_N} \le 2 \epsilon < 1
  \end{displaymath}
  Dada la contradicción, concluimos que \(L^{\infty}\) no puede ser separable. 
\end{Solucion}

% Problema 3
\begin{Problema}
  Suponga que el espacio de medida \((\Omega, M, \mu)\) es \(\sigma\)-finito. Decimos
  que una sucesión \(f_n \in L^p\) \textit{converge débilmente} a \(f\in L^p\) si
  \(c(f_n) \to c(f)\) para todo \(c\in (L^p)^{\ast}\). Escribimos \(f_n \rightharpoonup f\)
  en \(L^p\).
  \begin{enumerate}[(a)]
    \item Demuestre que \(f_n \rightharpoonup f\) en \(L^p\), \(p\in[1,\infty)\), si y solo
    si 
    \begin{displaymath}
      \int f_n g \to \int fg 
    \end{displaymath}
    para toda \(g\in L^{q}\), con \(1/p + 1/q = 1\).  

    \item 
    Pruebe que cuando \(f_n \rightharpoonup f\) en \(L^p\), \(\norm{f}_{p} \le \liminf_{n\to\infty} \norm{f_n}_{p}  \)   

    \item
    (Compacidad débil de \(L^p\)) Sea \(p\in (1,\infty)\) y suponga que \(L^q\) es separable.
    Pruebe que si \(\sup_{n} \norm{f_n}_{p} < \infty\), entonces existe \(f\in L^{p}\) y una
    sucesión \(f_{n_k} \in L^p\) tal que \(f_{n_k} \rightharpoonup f\).  

    \item De un contraejemplo del ítem anterior cuando \(p=1\). 
  \end{enumerate}
\end{Problema}
\begin{Solucion}
\begin{enumerate}[(a)]
  \item 
  \framebox{\(\Rightarrow\):} Notamos que para todo \(g\in L^{q}\), el mapa
  \(\Phi\colon L^{p} \to \K\) dado por \(a \mapsto \int a g\) define un funcional lineal acotado.
  En efecto, la linealidad es directa por la linealidad de la integral y la cota sale por Hölder:
  \begin{displaymath}
    \abs{ \Phi(a) }
    \le 
    \int \abs{ a g }
    \le
    \norm{a}_{p} \norm{g}_{q} < \infty.
  \end{displaymath}
  Luego, la convergencia débil nos da que
  \begin{displaymath}
    \lim_{n\to\infty} \Phi( f_n ) = \Phi(f)
    \Rightarrow
    \lim_{n\to\infty} \int f_n g = \int fg.
  \end{displaymath}
  \par\framebox{\(\Leftarrow\):} Por el teorema de representación de Riesz para espacios
  de funciones integrables, para todo \(T \in (L^p)^{\ast}\) existe \(h\ge 0\) en \(L^q\) tal que
  \begin{displaymath}
    T(a) = \int a h.
  \end{displaymath}
  Por la hipótesis, se sigue que
  \begin{displaymath}
    \lim_{n\to\infty} T(f_n) 
    = \lim_{n\to\infty} \int f_n h
    = \int f h
    = T(f).
  \end{displaymath}

  \item
  Sea \(x' \in (L^{p})^{\ast}\) de norma \(1\) tal que \(\abs{x'(f)} = \norm{f}_{p}\). 
  Esto lo podemos pedir por Hahn-Banach. 
  Por hipótesis, se cumple que \(x'(f_n) \to x'(f)\). Se sigue que:
  \begin{displaymath}
    \norm{f}_{p} 
    = \abs{x'(f)}
    = \abs{\lim_{n\to\infty} x'(f_n)}
    \le \liminf_{n\to\infty} \abs{x'(f_n)}
    \le \liminf_{n\to\infty} \norm{x'} \norm{f_n}_{p}
    = \liminf_{n\to\infty} \norm{f_n}_{p}.
  \end{displaymath}

  \item
  Recordemos que para \(1< p < \infty\), \(L^{q} \cong (L^p)^{\ast}\). Luego,
  \((L^{p})^{\ast}\) es separable. Tomemos \((\phi_n)_{n\in\N}\) en \((L^p)^{\ast}\) 
  denso y numerable. Luego, \((\phi_1(f_n))_n\) define una sucesión acotada de números 
  en el cuerpo. Se sigue que existe una subsecuencia \(f_{n,1}\) tal que
  \(\phi_1(f_{n,1}) \to c_1 \in \K\). De manera inductiva tenemos
  secuencias \((f_{n,k})_{n} \subset (f_{n,k-1})_n\) tal que 
  \begin{displaymath}
    \phi_{k}(f_{n,k}) \to c_k \in \K.
  \end{displaymath}
  Definamos la secuencia diagonal \(f_{n_k} = f_{n_k, n_k}\). Luego, para todo \(n\in\N\): 
  \begin{displaymath}
    \phi_n(f_{n_k}) \xrightarrow{n_k\to\infty} c_n \in \K.
  \end{displaymath}
  Definamos \(f = \lim f_{n_k}\). Notar que \(\phi_n(f_{n_k}) \to \phi_n(f)\). Sea
  \(g\in (L^{p})^{\ast}\). Luego,
  \begin{displaymath}
    \abs{g(f) - g(f_{n_k})}
    \le
    \abs{g(f) - \phi_n(f)}
    +
    \abs{\phi_n(f_{n_k} - \phi_{n}(f)}
    +
    \abs{\phi_{n}(f_{n_k}) - g(f_{n_k})}
    \xrightarrow{n,n_k\to\infty} 0.
  \end{displaymath}
\end{enumerate}
\end{Solucion}

% Problema 4
\begin{Problema}
\end{Problema}

% Problema 5
\begin{Problema}
  Sea \(\mu_i,\nu_i\) medidas \(\sigma\)-finitas en \((\Omega_i, M_i)\) tales que
  \(\nu_i \ll \mu_i\) para \(i=1,2\). Pruebe que la medida producto
  \(\nu_1\times\nu_2 \ll \mu_1\times\mu_2\) y que la derivada de Radon-Nikodým:
  \begin{displaymath}
    \left[ 
    \frac{d(\nu_1\times\nu_2)}{d(\mu_1\times\mu_2)}
    \right]
    =
    \left[ \frac{d\nu_1}{d\mu_1} \right](x_1)
    \left[ \frac{d\nu_2}{d\mu_2} \right](x_2)
  \end{displaymath}
\end{Problema}
\begin{Solucion}
  Denotemos por \(h_i \coloneqq [d\nu_i/d\mu_i]\) y por 
  \(H = [d(\nu_1\times\nu_2)/d(\mu_1\times\mu_2)]\). Además, pongamos
  \(\Pi_{\alpha} = \alpha_1 \times \alpha_2\).

  \vspace{1em}
  \noindent\underline{\(\Pi_{\nu} \ll \Pi_{\mu}\):} 
  Sea \(E \in \Pi_{M}\) un medible en el producto.
  Supongamos que \(\Pi_{\mu} (E) = 0\). Denotemos por \(E^{\bullet}, E_{\bullet}\) a las secciones
  en \(\Omega_2\) y \(\Omega_1\), respectivamente.  
  Luego (usando que las medidas son \(\sigma\)-finitas):
  \begin{align*}
    0 
    = \Pi_{\mu}(E) 
    &= \int_{\Omega_1} \mu_2(E^{x_1}) \, d\mu_1(x_1)
    = \int_{\Omega_2} \mu_1(E_{x_2}) \, d\mu_2(x_2)
    \\&\implies
    \mu_1(E_{\bullet}) = 0 
    \lor
    \mu_2(E^{\bullet}) = 0 
    \\&\implies
    \nu_{1}(E_{\bullet}) = 0
    \lor
    \nu_{2}(E^{\bullet}) = 0
    \\&\implies
    \Pi_{\nu}(E) 
    = \int_{\Omega_1} \nu_2(E^{x_1}) \, d\nu_1(x_1)
    = \int_{\Omega_2} \nu_1(E_{x_2}) \, d\nu_2(x_2)
    = 0.
  \end{align*}

  \vspace{1em}
  \noindent\underline{\(H(x_1, x_2) = h_1(x_1) h_2(x_2)\):}
  Para toda \(f \in L^1(\Pi_{\mu})\) se tiene:
  \begin{equation}\label{p5:eq1}
    \int_{\Omega_1\times\Omega_2} f \, d\Pi_{\nu}
    =
    \int_{\Omega_1\times\Omega_2} f H \, d\Pi_{\mu}
  \end{equation}
  Por otro lado: 
  \begin{equation}\label{p5:eq2}
  \begin{aligned}
    \int_{\Omega_1\times\Omega_2} f \, d\Pi_{\nu}
    &\overset{\text{fubini}}{=} 
    \int_{\Omega_1}
      \left( 
       \int_{\Omega_2}
        f
       \, d\nu_2
      \right)
      \, d\nu_1
    \\&\overset{\text{RN}}{=} 
    \int_{\Omega_1} 
      \left(
        \int_{\Omega_2} 
          f h_2(x_2)
        \, d\mu_2(x_2) 
      \right) 
    \, h_1(x_1) d\mu_1(x_1)
    \\&= 
    \int_{\Omega_1} 
      \left(
        \int_{\Omega_2} f h_1(x_1) h_2(x_2)
        \, d\mu_2(x_2)
      \right)
    \, d\mu_1(x_1) 
    \\&\overset{\text{fubini}}{=} 
    \int_{\Omega_1\times\Omega_2} f h_1(x_1) h_2(x_1) \, d\Pi_{\mu}
  \end{aligned}
  \end{equation}
  Restando~\eqref{p5:eq1} y~\eqref{p5:eq2} nos da que:
  \begin{displaymath}
    H(x_1,x_2) = h_1(x_1) h_2(x_2) 
    \quad\text{c.t.p. en } \Omega_1\times\Omega_2. 
  \end{displaymath}
\end{Solucion}

% Problema 6
\begin{Problema}
\end{Problema}

% Problema 7
\begin{Problema}
  Sea \(x_0(t) \in X = C([0,1])\) una función continua fija (\(\norm{x_0}_{\infty} = 1\))
  y \(L = Gen(x_0)\). Defina \(L\) el funcional lineal:
  \begin{displaymath}
    f(\lambda x_0) \coloneqq \lambda.
  \end{displaymath}
  \begin{enumerate}[(a)]
    \item Pruebe que \(\norm{f}_{L^{\ast}} = 1\).
    \item
    De acuerdo con el Teorema de Hahn-Banach, \(f\) se puede extender a un funcional lineal
    \(F\in X^{\ast}\) con norma \(\norm{F}_{X^{\ast}} = 1\). ¿Es la extensión única en 
    los siguientes casos?
      \begin{itemize}[topsep=0pt]
      \item \(x_0(t) = t\).
      \item \(x_0(t) = 1-2t\). 
    \end{itemize}
  \end{enumerate}
\end{Problema}
\begin{Solucion}
\begin{enumerate}[(a)]
  \item Apliquemos la definición:
  \begin{displaymath}
    \norm{f}_{L^{\ast}}
    =
    \sup_{\norm{\lambda x_0} = 1} \abs{f(\lambda x)}
    =
    \sup_{\abs{\lambda} = 1}
    \abs{\lambda}
    = 1.
  \end{displaymath}

  \item
  \begin{itemize}
    \item \(Gen(x_0) = \left\{ \lambda t \colon t\in [0,1] \right\}\). El mapa \(f\) entrega
    las pendientes de las rectas. Sea \(F\in X^{\ast}\) una extensión. Notemos que
    \(f\) está dominada por \(x(x-2)\).  

  \end{itemize}
\end{enumerate}
\end{Solucion}

% Problema 8
\begin{Problema}
  Suponga que \(X\) es un espacio de Banach.
  \begin{enumerate}[(a)]
    \item Pruebe que \(X\) es reflexivo si y solo si \(X^{\ast}\) es reflexivo.  
    \item Demuestre que si \(X^{\ast}\) es separable,  entonces \(X\) es separable. 
    (Sugerencia: Para cada \(n\in\N\) , escoja \(x_n\in X\) con \(\norm{x_n} = 1\) y 
    \(\abs{f_n(x_n)} \ge \frac{1}{2} \norm{f_n}\), donde \(f_n \in X^{\ast}\) es un subconjunto contable denso,
    y demuestre (por contradicción) que \(Gen_{\K_{c}}(\left\{ x_n \right\}_{n}) = X\), donde 
    \(Gen_{\K_c}(S)\) denota el conjunto de combinaciones lineales finitas de \(S\), con coeficientes en 
    \(\K_{c} = \Q\) cuando \(\K=\R\) y \(\K_c = \Q + i \Q\) cuando \(\K = \C\). 
    Note que la combinación de esta proposición y la Pregunta 2 muestra que, en general, 
    \((L^{\infty})^{\ast} \not\cong L^1\).
  \end{enumerate}
\end{Problema}
\begin{Solucion}
  Denotemos \(X^{(0)} \coloneqq X\) y \(X^{(i)} \coloneqq (X^{(i-1)})^{\ast}\) para \(i>0\).   
  \begin{enumerate}[(a)]
    \item 
    \framebox{\(\implies\)}: Supongamos que \(X^{(0)}\) es reflexivo. Entonces, el mapa
    \begin{alignat*}{1}
      J^{(0)} \colon X^{(0)} &\to X^{(2)}\\
      x &\mapsto 
      \begin{aligned}
        ev_{x} \colon X^{(1)} &\to \K\\
        x^{(1)} &\mapsto x^{(1)}(x)
      \end{aligned}.
    \end{alignat*}
  es un isomorfismo isométrico. Queremos probar lo mismo para el mapa:
  \begin{alignat*}{1}
    J^{(1)} \colon X^{(1)} &\to X^{(3)}\\
    x^{(1)} &\mapsto
    \begin{aligned}
      ev_{x^{(1)}} \colon X^{(2)} &\to \K\\
      x^{(2)} &\mapsto x^{(2)}(x^{(1)}) 
    \end{aligned}.
  \end{alignat*}
  Para esto, basta probar que es sobreyectivo. 

  Dado que \(J^{(0)}\) es sobreyectivo, todo elemento \(x_0^{(2)} \in X^{(2)}\) se
  puede escribir como \(J^{(0)}(x_0)\) para algún \(x_0 \in X\).
  Sea \(x^{(3)} \in X^{(3)}\) se tiene que 
  \begin{displaymath}
    x^{(3)}(x_0^{(2)})
    = x^{(3)} ( J^{(0)}(x_0) )
    = (x^{(3)}\circ J^{(0)}) (x_0)
  \end{displaymath}
  Denotemos por \(x^{(1)} \coloneqq x^{(3)} \circ J^{(0)} \in X^{(1)}\). Luego,
  \begin{displaymath}
    x^{(3)}(x_0^{(2)}) = x^{(1)}(x_0) = J^{(0)}(x_0)(x^{(1)}) = x_0^{(2)}(x^{(1)}).
  \end{displaymath}
  Y el último término lo podemos expresar como \(J^{(1)}(x^{(1)})(x_0^{(2)})\).
  Esto demuestra la sobreyectividad.
  \framebox{\(\impliedby\)}: 

  \item
  Supongamos \(X^{\ast}\) separable. Consideremos la esfera unitaria en \(X^{\ast}\) 
  \begin{displaymath}
    S^{(1)} \coloneqq \left\{ \norm{x^{(1)}} = 1 \right\}.
  \end{displaymath}
  Luego, \(S^{(1)}\) también es separable. Sean \(\left\{ x^{(1)}_n \right\}_{n\in\N}\)
  elementos densos y numerables. Asociemos, para cada \(n\), un elemento
  \(x_n \in X\) tal que \(0 < \delta \le \abs{x^{(2)}_n (x_n)} \le 1\). Probaremos que 
  esos \(x_n\) son densos (en el sentido del espacio generado) en \(X\). 
  Buscando una contradicción, supongamos que no lo son. Entonces existe un \(x^{(1)}\in S^{(1)}\)
  tal que \(x^{(1)}\) se anula en todo \(U = Gen_{\K_{c}}(\left\{ x_n \right\} )\) pero no es nulo
  fuera de \(U\). 
  (equivalentemente, existe un abierto en \(X\setminus U\) y podemos definir un funcional que se
  anule en todo menos ese abierto. Reescalando lo podemos pedir unitario).
  Por densidad en \(S^{(1)}\), existe \(N\) tal que \(\norm{x^{(1)}-x_N^{(1)}} > \delta/2\).
  Luego,
  \begin{displaymath}
    \delta \le \abs{x_N^{(1)}(x_N)} = \norm{x_N^{(1)}(x_N) - x^{(1)}(x_N)}
    \le \delta/2.
  \end{displaymath}
  Dada la contradicción, concluimos que \(x^{(1)}\) es nulo en \(X\) y por lo tanto
  los \(Gen_{\K_c}(\left\{ x_n \right\})\) es denso y numerable en \(X\).

  La numerabilidad viene de la numerabilidad de los coeficientes y las sumas finitas del generado.
  La densidad viene de la densidad de los coeficientes y lo que acabamos de probar.
\end{enumerate}
\end{Solucion}

\end{document}
