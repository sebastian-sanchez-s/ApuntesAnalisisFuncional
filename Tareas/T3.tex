\documentclass[11pt]{article}

% Page set up
\usepackage[margin=3cm]{geometry}

% Document font and symbols
\usepackage{mathptmx} % Times New Roman
\usepackage{amsmath,amsfonts,amsthm,amssymb}
\usepackage{esint}

% Language formatting
\usepackage[utf8]{inputenc}
\usepackage[T1]{fontenc}
\usepackage[spanish, es-noshorthands]{babel}

% Graph and Drawings
\usepackage{tikz,tikz-cd,float}
\usepackage{wrapfig}
\usetikzlibrary{babel}
\usepackage{pgfplots}
\usepackage[framemethod=default]{mdframed}
\usepackage{framed}

% Tools
\usepackage{multicol}
\usepackage{array}
\usepackage{hyperref}
\usepackage{xcolor}
\usepackage{etoolbox,mathtools}
\usepackage[shortlabels]{enumitem}

% Mdframed template
\mdfsetup{innertopmargin=5pt,%
	frametitlealignment=\raggedright,%
	frametitlefont=\texttt,%
	linewidth=2pt,%
	topline=false,rightline=false, bottomline=false,%
	frametitleaboveskip=\dimexpr-1\ht\strutbox\relax,}

% Problem Env
\newcounter{ProblemCounter}
\newenvironment{Problema}[1][]{%
	\vspace{1em}
	\refstepcounter{ProblemCounter}
  \noindent\tikz{\draw (0,0)--(\linewidth, 0) node[midway,fill=white]{{{\texttt{PROBLEMA}}~\theProblemCounter}};}
	}{
    \newline
    \noindent\tikz{\draw (0,0) -- (\linewidth, 0);}
	}

% Solution Env
\newenvironment{Solucion}[1][]
{%
  \newline
	\noindent{\ttfamily SOLUCIÓN}~
}%
{%
	%\hfill\(\blacksquare\)
}

% Resize abs and norm
\DeclarePairedDelimiter{\abs}{\lvert}{\rvert}
\DeclarePairedDelimiter{\norm}{\|}{\|}
\makeatletter
\let\oldabs\abs
\def\abs{\@ifstar{\oldabs}{\oldabs*}}
\let\oldnorm\norm
\def\norm{\@ifstar{\oldnorm}{\oldnorm*}}
\makeatother

% Shortcuts
\newcommand{\K}{\mathbb{K}}
\newcommand{\R}{\mathbb{R}}
\newcommand{\N}{\mathbb{N}}
\newcommand{\Z}{\mathbb{Z}}
\newcommand{\T}{\mathbb{T}}

\DeclareMathOperator{\esssup}{esssup}

% Header
\newcommand{\header}[1]
{
\begin{minipage}{.15\textwidth}
	\includegraphics[width=\textwidth,height=\textheight,keepaspectratio]{LogoUC}
\end{minipage}
\hspace{1em}
\begin{minipage}{.75\textwidth}
	\vspace{2em}
	{\scshape
	Pontificia Universidad Católica de Chile\\
	Facultad de Matemáticas\\
	Docente: Nikola Kamburov\\
	Ayudante: Matías Díaz
	}
\end{minipage}

\medskip

\begin{center}
	{\textbf{MAT2555 {-} Análisis Funcional}} \\[1ex]
	{{\large Tarea #1 {-} Omar Neyra, Sebastián Sánchez}}
\end{center}

\medskip
}

\begin{document}

\header{3}

% Problema 1
\begin{Problema}
  Sea \((\Omega, M, \mu)\) un espacio de medida y suponga que \(f\in L^{p_0}(\mu)
  \cap L^{\infty}(\mu)\) para algún \(p_0 \in [1,\infty)\). Pruebe que
  \(f\in L^p\) para todo \(p\ge p_0\) y que 
  \begin{displaymath}
    \norm{f}_{\infty} = \lim_{p\to\infty} \norm{f}_{p}.
  \end{displaymath}
\end{Problema}
\begin{Solucion}
  El caso \(p = p_0\) es directo, así que supongamos que la desigualdad es estricta y 
  denotemos \(p' \coloneqq p - p_0 > 0\). Notando que \(\abs{f(x)} \le \norm{f}_{\infty}\) 
  tenemos que
  \begin{equation}
    \int \abs{f}^{p} 
    =
    \int \abs{f}^{p_0} \abs{f}^{p'}
    \le
    \int \abs{f}^{p_0} \norm{f}_{\infty}^{p'}
    \le
    \norm{f}_{\infty}^{p'} \norm{f}_{p_0}^{p_0}
  \end{equation}
  Todas las cantidades son positivas, así que tomando raíz obtenemos que
  \begin{equation}
    \norm{f}_{p} \le \norm{f}_{\infty}^{p'/p} \norm{f}_{p_0}^{p_0/p} < \infty,
  \end{equation}
  pues la norma uniforme y \(p_0\) están acotadas. Tomando límite se ve directamente que
  \begin{equation}
    \lim_{p\to\infty} \norm{f}_{p} \le \norm{f}_{\infty}
  \end{equation}
  pues \(p'/p = 1 - p_0/p \to 1\) y \(p_0/p \to 0\) cuando \(p\to \infty\).   
\end{Solucion}

% Problema 2
\begin{Problema}
  Para todo \(a\in\R\) construya una función \(f_a \in L^{\infty}(\R)\) con
  \(\norm{f_a - f_b}_{L^{\infty}(\R)} \ge 1\) cuando \(a\ne b\). Demuestre que esto implica
  que \(L^{\infty}(\R)\) no es separable. 
\end{Problema}
\begin{Solucion}
  Consideremos \(f_a(x) = \sin(a x)\). Como \(-1 \le \sin \le 1\), \(f_a \in L^{\infty}\)
  para todo \(a\). Por otro lado, \(f_a \ge \frac{\sqrt{2}}{2}\) para
  \begin{equation}\label{p2:1}
    x\in \frac{1}{a} \left( \frac{\pi}{2} + 2\pi k - \frac{\pi}{4}, \frac{\pi}{2} + 2\pi k + \frac{\pi}{4} \right)
    \text{, con } \(k\in\Z\). 
  \end{equation}
  Similarmente, \(f_a \le - \frac{\sqrt{2}}{2}\) cuando
  \begin{equation}\label{p2:2}
    x\in \frac{1}{a} \left( \frac{3\pi}{2} + 2\pi k - \frac{\pi}{4}, \frac{3\pi}{2} + 2\pi k + \frac{\pi}{4} \right)
    \text{, con } \(k\in\Z\). 
  \end{equation}
  Llamemos a los intervalos \(I^a\) y \(I_{a}\) de las Ecuaciones~\eqref{p2:1} y~\eqref{p2:2},
  respectivamente.
\end{Solucion}

% Problema 3
\begin{Problema}
  Suponga que el espacio de medida \((\Omega, M, \mu)\) es \(\sigma\)-finito. Decimos
  que una sucesión \(f_n \in L^p\) \textit{converge débilmente} a \(f\in L^p\) si
  \(c(f_n) \to c(f)\) para todo \(c\in (L^p)^{\ast}\). Escribimos \(f_n \rightharpoonup f\)
  en \(L^p\).
  \begin{enumerate}[(a)]
    \item Demuestre que \(f_n \rightharpoonup f\) en \(L^p\), \(p\in[1,\infty)\), si y solo
    si 
    \begin{displaymath}
      \int f_n g \to \int fg 
    \end{displaymath}
    para toda \(g\in L^{q}\), con \(1/p + 1/q = 1\).  

    \item 
    Pruebe que cuando \(f_n \rightharpoonup f\) en \(L^p\), \(\norm{f}_{p} \le \liminf_{n\to\infty} \norm{f_n}_{p}  \)   

    \item
    (Compacidad débil de \(L^p\)) Sea \(p\in (1,\infty)\) y suponga que \(L^q\) es separable.
    Pruebe que si \(\sup_{n} \norm{f_n}_{p} < \infty\), entonces existe \(f\in L^{p}\) y una
    sucesión \(f_{n_k} \in L^p\) tal que \(f_{n_k} \rightharpoonup f\).  

    \item De un contraejemplo del ítem anterior cuando \(p=1\). 
  \end{enumerate}
\end{Problema}
\begin{Solucion}
\begin{enumerate}[(a)]
  \item 
  \framebox{\(\Rightarrow\):} Notamos que para todo \(g\in L^{q}\), el mapa
  \(\Phi\colon L^{p} \to \K\) dado por \(a \mapsto \int a g\) define un funcional lineal acotado.
  En efecto, la linealidad es directa por la linealidad de la integral y la cota sale por Hölder:
  \begin{displaymath}
    \abs{ \Phi(a) }
    \le 
    \int \abs{ a g }
    \le
    \norm{a}_{p} \norm{g}_{q} < \infty.
  \end{displaymath}
  Luego, la convergencia débil nos da que
  \begin{displaymath}
    \lim_{n\to\infty} \Phi( f_n ) = \Phi(f)
    \Rightarrow
    \lim_{n\to\infty} \int f_n g = \int fg.
  \end{displaymath}
  \par\framebox{\(\Leftarrow\):} Por el teorema de representación de Riesz para espacios
  de funciones integrables, para todo \(T \in (L^p)^{\ast}\) existe \(h\ge 0\) en \(L^q\) tal que
  \begin{displaymath}
    T(a) = \int a h.
  \end{displaymath}
  Por la hipótesis, se sigue que
  \begin{displaymath}
    \lim_{n\to\infty} T(f_n) 
    = \lim_{n\to\infty} \int f_n h
    = \int f h
    = T(f).
  \end{displaymath}

  \item
  Usaremos el lema de Fatou con \(F_n = \abs{f_n}^p\). Como \(f_n\in L^p\), se tiene que
  \(F_n \in L^1\), además, es evidente que \(F_n \ge 0\). Definamos
  \begin{displaymath}
    F = \liminf_{n\to\infty} F_n < \infty\qquad c.t.p.
  \end{displaymath}
  Luego,
  \begin{displaymath}
    \int F \le \liminf_{n\to\infty} \int F_n = \liminf_{n\to\infty} \norm{f_n}_{p}^{p}.
  \end{displaymath}
  Como \(\int\) es un funcional lineal acotado, la convergencia débil implica que
  \begin{displaymath}
    \int f_n \to \int f
    \implies
    \int F_n \to \int F.
  \end{displaymath}
  En particular, lo hace en \(\liminf\). Así, nos queda que
  \begin{displaymath}
    \norm{f}_{p}^{p} \le \liminf_{n\to\infty} \norm{f_n}_{p}^{p}.
  \end{displaymath}
\end{enumerate}
\end{Solucion}

\end{document}
