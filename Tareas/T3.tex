\documentclass[11pt]{article}

% Page set up
\usepackage[margin=3cm]{geometry}

% Document font and symbols
\usepackage{mathptmx} % Times New Roman
\usepackage{amsmath,amsfonts,amsthm,amssymb}
\usepackage{esint}
\usepackage{bbm}

% Language formatting
\usepackage[utf8]{inputenc}
\usepackage[T1]{fontenc}
\usepackage[spanish, es-noshorthands]{babel}

% Graph and Drawings
\usepackage{tikz,tikz-cd,float}
\usepackage{wrapfig}
\usetikzlibrary{babel}
\usepackage{pgfplots}
\usepackage[framemethod=default]{mdframed}
\usepackage{framed}

% Tools
\usepackage{multicol}
\usepackage{array}
\usepackage{hyperref}
\usepackage{xcolor}
\usepackage{etoolbox,mathtools}
\usepackage[shortlabels]{enumitem}

% Mdframed template
\mdfsetup{innertopmargin=5pt,%
	frametitlealignment=\raggedright,%
	frametitlefont=\texttt,%
	linewidth=2pt,%
	topline=false,rightline=false, bottomline=false,%
	frametitleaboveskip=\dimexpr-1\ht\strutbox\relax,}

% Problem Env
\newcounter{ProblemCounter}
\newenvironment{Problema}[1][]{%
	\vspace{1em}
	\refstepcounter{ProblemCounter}
  \noindent\tikz{\draw (0,0)--(\linewidth, 0) node[midway,fill=white]{{{\texttt{PROBLEMA}}~\theProblemCounter}};}
	}{
    \ \newline
    \noindent\tikz{\draw (0,0) -- (\linewidth, 0);}
	}

% Solution Env
\newenvironment{Solucion}[1][]
{%
  \newline
	\noindent{\ttfamily SOLUCIÓN}~
}%
{%
	%\hfill\(\blacksquare\)
}

% Resize abs and norm
\newcommand\myeq[1]{\mathrel{\overset{\makebox[0pt]{\mbox{\normalfont\tiny\sffamily #1}}}{=}}}
\DeclarePairedDelimiter{\abs}{\lvert}{\rvert}
\DeclarePairedDelimiter{\norm}{\|}{\|}
\DeclarePairedDelimiter\autobracket{(}{)}
\newcommand{\br}[1]{\autobracket*{#1}}
\newcommand{\lbr}[1]{\left\langle #1\right\rangle}
\newcommand{\set}[1]{\left\lbrace #1\right\rbrace}
\makeatletter
\let\oldabs\abs
\def\abs{\@ifstar{\oldabs}{\oldabs*}}
\let\oldnorm\norm
\def\norm{\@ifstar{\oldnorm}{\oldnorm*}}
\makeatother

% Shortcuts
\newcommand{\K}{\mathbb{K}}
\newcommand{\Q}{\mathbb{Q}}
\newcommand{\C}{\mathbb{C}}
\newcommand{\R}{\mathbb{R}}
\newcommand{\N}{\mathbb{N}}
\newcommand{\Z}{\mathbb{Z}}
\newcommand{\T}{\mathbb{T}}

\DeclareMathOperator{\esssup}{esssup}
\DeclareMathOperator{\Gen}{Gen}

% Header
\newcommand{\header}[1]
{
\begin{minipage}{.15\textwidth}
	\includegraphics[width=\textwidth,height=\textheight,keepaspectratio]{LogoUC}
\end{minipage}
\hspace{1em}
\begin{minipage}{.75\textwidth}
	\vspace{2em}
	{\scshape
	Pontificia Universidad Católica de Chile\\
	Facultad de Matemáticas\\
	Docente: Nikola Kamburov\\
	Ayudante: Matías Díaz
	}
\end{minipage}

\medskip

\begin{center}
	{\textbf{MAT2555 {-} Análisis Funcional}} \\[1ex]
	{{\large Tarea #1 {-} Omar Neyra, Sebastián Sánchez}}
\end{center}

\medskip
}

\begin{document}

\header{3}

% Problema 1
\begin{Problema}
  Sea \((\Omega, M, \mu)\) un espacio de medida y suponga que \(f\in L^{p_0}(\mu)
  \cap L^{\infty}(\mu)\) para algún \(p_0 \in [1,\infty)\). Pruebe que
  \(f\in L^p\) para todo \(p\ge p_0\) y que 
  \begin{displaymath}
    \norm{f}_{\infty} = \lim_{p\to\infty} \norm{f}_{p}.
  \end{displaymath}
\end{Problema}
\begin{Solucion}
  El caso \(p = p_0\) es directo, así que supongamos que la desigualdad es estricta y 
  denotemos \(p' \coloneqq p - p_0 > 0\). Notando que \(\abs{f(x)} \le \norm{f}_{\infty}\) 
  tenemos que
  \begin{equation}
    \int \abs{f}^{p} 
    =
    \int \abs{f}^{p_0} \abs{f}^{p'}
    \le
    \int \abs{f}^{p_0} \norm{f}_{\infty}^{p'}
    \le
    \norm{f}_{\infty}^{p'} \norm{f}_{p_0}^{p_0}
  \end{equation}
  Todas las cantidades son positivas, así que tomando raíz obtenemos que
  \begin{equation}
    \norm{f}_{p} \le \norm{f}_{\infty}^{p'/p} \norm{f}_{p_0}^{p_0/p} < \infty,
  \end{equation}
  pues la norma uniforme y \(p_0\) están acotadas. Tomando límite se ve directamente que
  \begin{equation}
    \lim_{p\to\infty} \norm{f}_{p} \le \norm{f}_{\infty}
  \end{equation}
  pues \(p'/p = 1 - p_0/p \to 1\) y \(p_0/p \to 0\) cuando \(p\to \infty\).

   Para la otra dirección, consideremos \(\epsilon > 0\). Luego,
  \begin{alignat*}{2}
    \norm{f}_{p} 
    &= \left( \int \abs{f}^p \right)^{1/p}
    &&\ge
    \left( \int_{\left\{ \abs{f} + \epsilon > \norm{f}_{\infty} \right\} } \abs{f}^{p} \right)^{1/p}
    \\&\ge
    \left( \int_{\left\{ \abs{f} + \epsilon > \norm{f}_{\infty} \right\} } (\norm{f}_{\infty} - \epsilon)^{p} \right)^{1/p}
    &&\ge
    (\norm{f}_{\infty}-\epsilon) \mu( \left\{ \abs{f} + \epsilon > \norm{f}_{\infty} \right\} )^{1/p}.
  \end{alignat*}
  Notar que \(\mu(\left\{ \abs{f} + \epsilon > \norm{f}_{\infty} \right\}) < \infty\) pues \(f\in L^p\).
  Tomando límite tenemos que \(\lim_{p\to\infty} \norm{f}_p \ge \norm{f}_{\infty} - \epsilon\).
  Como \(\epsilon\) es arbitrario, se concluye el resultado. 

\end{Solucion}

\newpage
% Problema 2
\begin{Problema}
  Para todo \(a\in\R\) construya una función \(f_a \in L^{\infty}(\R)\) con
  \(\norm{f_a - f_b}_{L^{\infty}(\R)} \ge 1\) cuando \(a\ne b\). Demuestre que esto implica
  que \(L^{\infty}(\R)\) no es separable. 
\end{Problema}
\begin{Solucion}
  Para cada \(a\in\R\) definamos \(f_a(x)=\mathbbm{1}_{[a,\infty)}(x)\). Claramente \(f_a\in L^\infty(\R)\) para todo \(a\in \R\). Sea \(b\neq a\in\R\). Supongamos sin perdida de generalidad que \(a<b\). Luego, \((a,b)\neq\varnothing\) y por lo tanto \((f_a-f_b)(x)=1\) para todo \(x\in (a,b)\). Concluimos que, ya que \((a,b)\) es un conjunto de medida no nula, \(\norm{f_a-f_b}_{\infty}\geq 1\).

  Supongamos por contradicción que \(L^\infty(\R)\) es separable. Sea \(U:=\set{u_n}_{n\in\N}\subseteq L^\infty(\R)\) denso. Definamos \(B_a:=B_{1/2}(f_a)\subseteq L^{\infty}(\R)\). Notemos que la intersección
  \begin{displaymath}
      B_a\cap U
  \end{displaymath}
  es no vacía para a lo más numerables valores de \(a\). Luego, el conjunto \(S:=\set{a\in \R: B_a\cap U=\varnothing}\) es no vacio. Se tiene enconces que el conjunto \(\bigcup_{a\in S}B_a\) es un conjunto abierto y luego
  \begin{displaymath}
      \norm{f_b-u_n}_\infty \geq \dfrac{1}{2} \;\forall b\in S, n\in\N.
  \end{displaymath}
  Esto contradice la densidad de \(U\), por lo que tenemos lo pedido.
\end{Solucion}

\newpage
% Problema 3
\begin{Problema}
  Suponga que el espacio de medida \((\Omega, M, \mu)\) es \(\sigma\)-finito. Decimos
  que una sucesión \(f_n \in L^p\) \textit{converge débilmente} a \(f\in L^p\) si
  \(c(f_n) \to c(f)\) para todo \(c\in (L^p)^{\ast}\). Escribimos \(f_n \rightharpoonup f\)
  en \(L^p\).
  \begin{enumerate}[(a)]
    \item Demuestre que \(f_n \rightharpoonup f\) en \(L^p\), \(p\in[1,\infty)\), si y solo
    si 
    \begin{displaymath}
      \int f_n g \to \int fg 
    \end{displaymath}
    para toda \(g\in L^{q}\), con \(1/p + 1/q = 1\).  

    \item 
    Pruebe que cuando \(f_n \rightharpoonup f\) en \(L^p\), \(\norm{f}_{p} \le \liminf_{n\to\infty} \norm{f_n}_{p}  \)   

    \item
    (Compacidad débil de \(L^p\)) Sea \(p\in (1,\infty)\) y suponga que \(L^q\) es separable.
    Pruebe que si \(\sup_{n} \norm{f_n}_{p} < \infty\), entonces existe \(f\in L^{p}\) y una
    sucesión \(f_{n_k} \in L^p\) tal que \(f_{n_k} \rightharpoonup f\).  

    \item De un contraejemplo del ítem anterior cuando \(p=1\). 
  \end{enumerate}
\end{Problema}
\begin{Solucion}
\begin{enumerate}[(a)]
  \item 
  \framebox{\(\Rightarrow\):} Notamos que para todo \(g\in L^{q}\), el mapa
  \(\Phi\colon L^{p} \to \K\) dado por \(a \mapsto \int a g\) define un funcional lineal acotado.
  En efecto, la linealidad es directa por la linealidad de la integral y la cota sale por Hölder:
  \begin{displaymath}
    \abs{ \Phi(a) }
    \le 
    \int \abs{ a g }
    \le
    \norm{a}_{p} \norm{g}_{q} < \infty.
  \end{displaymath}
  Luego, la convergencia débil nos da que
  \begin{displaymath}
    \lim_{n\to\infty} \Phi( f_n ) = \Phi(f)
    \Rightarrow
    \lim_{n\to\infty} \int f_n g = \int fg.
  \end{displaymath}
  \par\framebox{\(\Leftarrow\):} Por el teorema de representación de Riesz para espacios
  de funciones integrables, para todo \(T \in (L^p)^{\ast}\) existe \(h\ge 0\) en \(L^q\) tal que
  \begin{displaymath}
    T(a) = \int a h.
  \end{displaymath}
  Por la hipótesis, se sigue que
  \begin{displaymath}
    \lim_{n\to\infty} T(f_n) 
    = \lim_{n\to\infty} \int f_n h
    = \int f h
    = T(f).
  \end{displaymath}

  \item
  Sea \(x' \in (L^{p})^{\ast}\) de norma \(1\) tal que \(\abs{x'(f)} = \norm{f}_{p}\). 
  Esto lo podemos pedir por Hahn-Banach. 
  Por hipótesis, se cumple que \(x'(f_n) \to x'(f)\). Se sigue que:
  \begin{displaymath}
    \norm{f}_{p} 
    = \abs{x'(f)}
    = \abs{\lim_{n\to\infty} x'(f_n)}
    \le \liminf_{n\to\infty} \abs{x'(f_n)}
    \le \liminf_{n\to\infty} \norm{x'} \norm{f_n}_{p}
    = \liminf_{n\to\infty} \norm{f_n}_{p}.
  \end{displaymath}

  \item 
  Recordemos que para \(1< p < \infty\), \(L^{q} \cong (L^p)^{\ast}\). Luego,
  \((L^{p})^{\ast}\) es separable. Tomemos \((\phi_n)_{n\in\N}\) en \((L^p)^{\ast}\) 
  denso y numerable. Luego, \((\phi_1(f_n))_n\) define una sucesión acotada de números 
  en el cuerpo. Se sigue que existe una subsecuencia \(f_{n,1}\) tal que
  \(\phi_1(f_{n,1}) \to c_1 \in \K\). De manera inductiva tenemos
  secuencias \((f_{n,k})_{n} \subset (f_{n,k-1})_n\) tal que 
  \begin{displaymath}
    \phi_{k}(f_{n,k}) \to c_k \in \K.
  \end{displaymath}
  Definamos la secuencia diagonal \(f_{n_k} = f_{n_k, n_k}\). Luego, para todo \(n\in\N\): 
  \begin{displaymath}
    \phi_n(f_{n_k}) \xrightarrow{n_k\to\infty} c_n \in \K.
  \end{displaymath}
  Definamos \(f = \lim f_{n_k}\). Notar que \(\phi_n(f_{n_k}) \to \phi_n(f)\).
  Sea \(\epsilon > 0\) y tomemos \(g\in (L^{p})^{\ast}\). Por densidad de las
  \(\phi_n\), existe un \(N_1 > 0\) tal que:
  \begin{displaymath}
      \norm{\phi_N - g}_{p^{\ast}} < \frac{\epsilon}{3}.
  \end{displaymath}
  Por otro lado, existe \(N_1\) tal que:
  \begin{displaymath}
      \abs{\phi_n( f_{n_k} ) - \phi_n(f)} < \frac{\epsilon}{3}
  \end{displaymath}
  para \(n_k > N_1\). Tomando \(N > \max(N_0, N_1)\) nos da que:
  \begin{displaymath}
    \abs{g(f) - g(f_{n_k})}
    \le
    \underbrace{\abs{g(f) - \phi_n(f)}}_{\le \norm{g-\phi_n}}
    +
    \abs{\phi_n(f_{n_k}) - \phi_{n}(f)}
    +
    \underbrace{\abs{\phi_{n}(f_{n_k}) - g(f_{n_k})}}_{\le \norm{g-\phi_n}}
    \le
    \epsilon.
  \end{displaymath}

  \item Sea \((f_n)_{n\in\N}\subseteq L^1([0,1])\) definidas por \(f_n(t)=n\mathbbm{1}_{[0,1/n]}(t)\). Claramente, para \(n\in\N\)
  \[\norm{f_n}_1=\int_{0}^1 \abs{f(t)}dt=\int_{0}^{1/n}\abs{n}dt=1.\]
  Luego, \(\sup_{n\in\N}\norm{f_n}_1<\infty\).
  Notemos que para toda \(g\in C([0,1])\subseteq L^{\infty}([0,1])\),
  \begin{equation*}
      \int_{0}^1 u_n(t)g(t)dt=n\int_{0}^{1/n}g(t)dx.
  \end{equation*}
  Como \(g\) es continua en \([0,1/n]\), este alcanza su maximo y su minimo en el intervalo. Luego,
  \begin{align*}
      \int_{0}^{1/n}g(t)dt\leq \dfrac{1}{n}\max_{t\in[0,1/n]}g(t);\\
      \int_{0}^{1/n}g(t)dt\geq \dfrac{1}{n}\min_{t\in[0,1/n]}g(t),
  \end{align*}
  y te tendría que \(\int_{0}^1 f_n(t)g(t)dt\to g(0)\).
  
  Sea \((f_{n_k})_k\) una subsucesión. Supongamos por contradicción que existe \(f\in L^1([0,1])\) tal que \(f_{n_k}\rightharpoonup f\). Se debe cumplir que
  \begin{equation*}
      \int_{0}^1f_{n_k}(x)\varphi(t)dt \to \int_{0}^1 f(t)\varphi(t)dt
  \end{equation*}
  para toda \(\varphi\in L^\infty([0,1])\). Por lo anterior, si tomamos \(\psi\in C([0,1])\),
  \begin{equation*}
      \int_{0}^1f_{n_k}(x)\psi(t)dt \to \psi(0),
  \end{equation*}
  y por unicidad del limite, \(f\) cumple que \(\int_{0}^1 f(t)\psi(t)dt=\psi(0)\) para cualquier \(\psi\) continua.

  Tomemos \((\psi_n)_{n\in\N}\subseteq C([0,1])\subseteq L^{\infty}([0,1])\) tales que \(\psi_n(t)=(1-t)^n\). Se tiene que \(\int_{0}^1f(t)\psi_n(t)dt=\psi_n(0)=1\) para cualquier \(n\in \N\), por lo que  \(\psi_n(0)\xrightarrow{n\to\infty} 1\). Notemos que para todo \(n\in\N\), \(\psi_n(t)\leq 1\), \(t\in [0,1]\) por lo que, como \(u\in L^1([0,1])\), por el TCD,
  \begin{align*}
      1&=\lim_{n\to\infty}\psi_n(0)\\
      &=\lim_{n\to\infty}\int_{0}^1f(t)\psi_n(t)dt\\
      &\myeq{TCD}\int_{0}^1f(t)\lim_{n\to\infty}\psi_n(t)dt.
  \end{align*}
  Como \(\psi_n\to \psi\), con \(\psi(t)=\mathbbm{1}_{\set{0}}(t)\) puntualmente,
  \[\int_{0}^1f(t)\lim_{n\to\infty}\psi_n(t)dt=0\].
  Esta contradicción demuestra que no existe tal \(f\), por lo que tenemos lo pedido.
\end{enumerate}
\end{Solucion}

\newpage
% Problema 4
\begin{Problema}
Sean \((\Omega_i, \mathcal{M}_i, \mu_i)\), \(i=1,2\) dos espacios de medida \(\sigma\)-finita y sea \(K:\Omega_1\times \Omega_2\to\K\) una función \(\mu_1\otimes \mu_2\) medible. Fije \(1\leq q\leq \infty\).
\begin{enumerate}[(a)]
    \item Suponga que \(K\geq 0\). Utilice la propiedad isométrica de la correspondencia entre \((L^p)^*\) y \(L^q\) para demostrar la \textit{Desigualdad integral de Minkowski}:
    \begin{displaymath}
        \norm{\int_{\Omega_2}K(\cdot, y)d\mu_2(y)}_{L^q(\Omega_1,\mu_1)}\leq \int_{\Omega_2}\norm{K(\cdot, y)}_{L^q(\Omega_1,\mu_1)}d\mu_2(y).
    \end{displaymath}
    \item Pruebe que la desigualdad se cumple también cuando \(K(\cdot, y)\in L^q(\Omega_1,\mu_1)\) para todo \(y\in\Omega_2\).
\end{enumerate}
\end{Problema}
\begin{Solucion}
\begin{enumerate}[(a)]
  \item
  Notemos que \(\mu_1\times\mu_2\) es \(\sigma\)-finita. 
  Para \(p=1\) basta aplicar Tonelli. Supongamos que \(p>1\). 
  Sea \(\Omega_{n} \uparrow \Omega_1\times\Omega_2\) tal que \(\mu_1\times\mu_2(\Omega_n) < \infty\). 
  Sea \(\Phi\) el mapa:
  \begin{align*}
    \Phi\colon L^q(\Omega_1\times\Omega_2) &\to (L^p(\Omega_1\times\Omega_2))^{\ast}\\
    g &\mapsto \langle \cdot, g \rangle
  \end{align*}
  donde \(g\in L^{q}(\Omega_1\times\Omega_2)\). Sabemos que \(\Phi\) es un isomorfismo isométrico
  porque \(p\in[1,\infty)\) y \(\mu_1\times\mu_2\) es \(\sigma\)-finito.

  Sea \(s_n \in \Omega_{n}\) una función simple. Notar que \(s_n \in L^{q}(\Omega_n)\) y
  \(\int s_n(\cdot,x_2) \in L^{q}(\Omega_n^{x_1}))\). Tomemos \(g\in L^{p}(\Omega_n^{x_1}\) de norma \(1\).
  Luego,
  \begin{align*}
    \abs{\Phi^{x_1}( \int s_n(\cdot,x_2) d\mu_2 )(g)}
    &=
    \abs{ \langle \int s_n(\cdot,x_2) , g \rangle}
    \\&\le
    \int_{\Omega_1} \int_{\Omega_2} \abs{ s_n(x_1,x_2) g(x_2) }
    \\&\overset{\text{tonelli}}{=}
    \int_{\Omega_2} \int_{\Omega_1} \abs{ s_n(x_1,x_2) g(x_2) }
    \\&\overset{\text{hölder}}{\le}
    \int_{\Omega_2} \norm{ s_n(\cdot, x_2) }_{q}
  \end{align*}
  Dado que \(\Phi\) es isometría, se tiene que
  \begin{displaymath}
    \norm{\Phi(\int s_n(\cdot,x_2) d\mu_2}_{p^{\ast}}
    =
    \norm{\int s_{n}(\cdot, x_2) d\mu_2}_{q}
  \end{displaymath}
  y se concluye el resultado. Dado que \(K\ge 0\) y es medible, existe una sucesión
  de funciones simples que convergen a \(K\) monotonamente. Combinando el procedimiento
  anterior concluimos que vale para \(K\).

  \item
  Aplicando el mapa \(\Phi\) sobre \(K\) se sigue lo pedido como se muestra arriba para
  funciones simples. Esto ya que el mapa \(\Phi\) es isometría entre \(L^{q}\) y \((L^p)^{\ast}\)
  y no importa el signo.
\end{enumerate}
\end{Solucion}

\newpage
% Problema 5
\begin{Problema}
  Sea \(\mu_i,\nu_i\) medidas \(\sigma\)-finitas en \((\Omega_i, M_i)\) tales que
  \(\nu_i \ll \mu_i\) para \(i=1,2\). Pruebe que la medida producto
  \(\nu_1\times\nu_2 \ll \mu_1\times\mu_2\) y que la derivada de Radon-Nikodým:
  \begin{displaymath}
    \left[ 
    \frac{d(\nu_1\times\nu_2)}{d(\mu_1\times\mu_2)}
    \right]
    =
    \left[ \frac{d\nu_1}{d\mu_1} \right](x_1)
    \left[ \frac{d\nu_2}{d\mu_2} \right](x_2)
  \end{displaymath}
\end{Problema}
\begin{Solucion}
  Denotemos por \(h_i \coloneqq [d\nu_i/d\mu_i]\) y por 
  \(H = [d(\nu_1\times\nu_2)/d(\mu_1\times\mu_2)]\). Además, pongamos
  \(\Pi_{\alpha} = \alpha_1 \times \alpha_2\).

    \vspace{1em}
  \noindent\underline{\(\Pi_{\nu} \ll \Pi_{\mu}\):} 
  Sea \(E \in \Pi_{M}\) un medible en el producto.
  Supongamos que \(\Pi_{\mu} (E) = 0\). Denotemos por \(E^{\bullet}, E_{\bullet}\) a las secciones
  en \(\Omega_2\) y \(\Omega_1\), respectivamente.  
  Luego (usando que las medidas son \(\sigma\)-finitas):
  \begin{align*}
    0 
    = \Pi_{\mu}(E) 
    &= \int_{\Omega_1} \mu_2(E^{x_1}) \, d\mu_1(x_1)
    = \int_{\Omega_2} \mu_1(E_{x_2}) \, d\mu_2(x_2)
    \\&\implies
    \mu_1(E_{\bullet}) = 0 
    \lor
    \mu_2(E^{\bullet}) = 0 
    \\&\implies
    \nu_{1}(E_{\bullet}) = 0
    \lor
    \nu_{2}(E^{\bullet}) = 0
    \\&\implies
    \Pi_{\nu}(E) 
    = \int_{\Omega_1} \nu_2(E^{x_1}) \, d\nu_1(x_1)
    = \int_{\Omega_2} \nu_1(E_{x_2}) \, d\nu_2(x_2)
    = 0.
  \end{align*}

  \vspace{1em}
  \noindent\underline{\(H(x_1, x_2) = h_1(x_1) h_2(x_2)\):}
  Para toda \(f \in L^1(\Pi_{\mu})\) se tiene:
  \begin{equation}\label{p5:eq1}
    \int_{\Omega_1\times\Omega_2} f \, d\Pi_{\nu}
    =
    \int_{\Omega_1\times\Omega_2} f H \, d\Pi_{\mu}
  \end{equation}
  Por otro lado: 
  \begin{equation}\label{p5:eq2}
  \begin{aligned}
    \int_{\Omega_1\times\Omega_2} f \, d\Pi_{\nu}
    &\overset{\text{fubini}}{=} 
    \int_{\Omega_1}
      \left( 
       \int_{\Omega_2}
        f
       \, d\nu_2
      \right)
      \, d\nu_1
    \\&\overset{\text{RN}}{=} 
    \int_{\Omega_1} 
      \left(
        \int_{\Omega_2} 
          f h_2(x_2)
        \, d\mu_2(x_2) 
      \right) 
    \, h_1(x_1) d\mu_1(x_1)
    \\&= 
    \int_{\Omega_1} 
      \left(
        \int_{\Omega_2} f h_1(x_1) h_2(x_2)
        \, d\mu_2(x_2)
      \right)
    \, d\mu_1(x_1) 
    \\&\overset{\text{fubini}}{=} 
    \int_{\Omega_1\times\Omega_2} f h_1(x_1) h_2(x_2) \, d\Pi_{\mu}
  \end{aligned}
  \end{equation}
  Restando~\eqref{p5:eq1} y~\eqref{p5:eq2} nos da que:
  \begin{displaymath}
    H(x_1,x_2) = h_1(x_1) h_2(x_2) 
    \quad\text{c.t.p. en } \Omega_1\times\Omega_2. 
  \end{displaymath}
\end{Solucion}

\newpage
% Problema 6
\begin{Problema}
Pruebe que si \(Y\) es un espacio vectorial de dimensión finita en un espacio normado \(X\), entonces \(Y\) tiene un \textit{complemento cerrado}, i.e. existe un subespacio cerrado \(Z\) tal que \(X=Y\oplus Z\).
\end{Problema}
\begin{Solucion}
    Sea \(e_1,\hdots,e_n\subset X\) una base de \(Y\). Podemos definir \(f_i\in Y^*\) tales que
    \begin{displaymath}
        f_i(e_j) = \delta_{ij}.
    \end{displaymath}
    Por Hahn-Banach, podemos extender cada \(f_i\) a una extensión  \(g_i\in X^*\). Definamos \(Z:=\bigcap_{i=1}^n \ker g_i\). Sea \(x\in X\). Tomemos \(y=\sum_{i=1}^{n}g_i(x)e_i\in Y\). Tenemos que para cada \(j=1,\hdots, n\),
    \begin{displaymath}
        g_j(y)=\sum_{i=1}^{n}g_j(e_i)g_i(x)=\sum_{i=1}^{n}f_j(e_i)g_i(x)=g_j(x),
    \end{displaymath}
    Por lo que \(g_j(x-y)=0\) para \(j=1,\hdots, n \Rightarrow x-y\in Z\). Como \(x\in X\) arbitrario, \(X=Y+Z\). Falta demostrar que \(Y\cap Z=\set{0}\). Esto es directo ya que si \(x\in Y\cap Z\), entonces
    \(x=\sum_{i=1}^n \alpha_i e_i\), \(\alpha_i\in\K\). Luego, ya que \(x\in Z\), para \(j=1,\hdots,n\),
    \begin{align*}
        0=f_j(x)=\sum_{i=1}^{n}\alpha_if_j(e_i)=\alpha_j.\Rightarrow x=0.
    \end{align*}
\end{Solucion}

\newpage
% Problema 7
\begin{Problema}
Sea \(x_o(t)\in X=C([0,1])\) una función continua fija y sea \(L=Gen(x_0)\). Defina en \(L\) el funcional lineal
\begin{displaymath}
    f(x):=\lambda \quad\text{si }x=\lambda x_0.
\end{displaymath}
\begin{enumerate}[(a)]
    \item Pruebe que \(\norm{f}_{L^*}=1\)
    \item De acuerdo con el teorema de Hahn-Banach, \(f\) se puede extender a un funcional \(F\in X^*\) con norma \(\norm{F}_{X^*}=1\). Es la extensión única si
    \begin{itemize}[\(\bullet\)]
        \item \(x_0(t)=t\);
        \item \(x_0(t)=1-2t\)?
    \end{itemize}
\end{enumerate}
\end{Problema}
\begin{Solucion}
    \begin{enumerate}[(a)]
        \item Se tiene que
        \begin{align*}
            \norm{f}_{L^*}&=\sup_{\norm{y}_\infty\neq 0}\dfrac{\abs{f(y)}}{\norm{y}}\\
            &=\sup_{\lambda\neq 0}\dfrac{\abs{f(\lambda x_0)}}{\norm{\lambda x_0}_\infty}\\
            &=\sup_{\lambda\neq 0}\dfrac{\abs{\lambda}}{\abs{\lambda}\norm{x_0}_\infty}=1.
        \end{align*}
        \item \begin{itemize}[\(\bullet\)]
            \item \fbox{\(x_0(t)=t\)} Veamos que el funcional de evaluación \(\Tilde{f}(x)=x(1)\in X^*\) es el único funcional lineal que extiende \(f\). Sea \(F\in X^*\) una extensión de \(f\). Sea \(\varphi\in X\) tal que \(\varphi(t)=0\) para \(t\in[1-\delta,1]\), para algún \(\delta>0\). Veamos que \(F(\varphi)=0\). Por contradicción, supongamos sin perdida de generalidad que \(F(\varphi)>0\). Definamos \(\psi_\varepsilon(t)=t+\varepsilon\varphi(t)\) (Si \(F(\varphi)<0\), lo definimos como \(\psi_{\varepsilon}(t)=t-\varepsilon\varphi(t)\)). Para \(\varepsilon\) suficientemente pequeño, \(\norm{\psi_\varepsilon}_\infty\leq 1\), pero por linealidad,
            \[\abs{F(\psi_\varepsilon)}=\abs{F(t)+\varepsilon F(\varphi)}>1,\]
            lo que contradice que \(\norm{F}_{X^*}=1\).
            
            Veamos ahora que para \(\varphi\in X\) tales que \(\varphi(1)=0\), se cumple que \(F(\varphi)=0\). Sea \((\varphi_n)_{n\in\N}\subseteq X\) tal que \(\varphi_n\to \varphi\) uniforme, y con \(\varphi_n(t)=0\) para \(t\in[1-\delta_n,1]\). Por continuidad de \(F\),
            \[F(\varphi)=\lim_{n\to\infty}F(\varphi_n)=0.\]
            Finalmente, se tiene que \(F(x)=x(1)\) para todo \(x\in X\) ya que definiendo \(\varphi(t)=x(t)-x(1)t\), como \(\varphi(1)=0\),
            \[0=F(\varphi)=F(x)-x(1)\Rightarrow F(x)=x(1).\]
            Como \(x\in X\) arbitrario, concluimos que \(F=\Tilde{f}\), que es lo que queríamos demostrar.
            \item \fbox{\(x_0(t)=1-2t\)} Basta notar que los funcionales de evaluación \(f_1,f_2\in X^*\) tales que \(f_1(x)=x(0)\), \(f_2(x)=-x(1)\) cumplen que para \(y=\lambda x_0\in L\),
            \begin{align*}
                f_1(y)&=y(0)=\lambda x_0(0)=\lambda,\\
                f_2(y)&=-y(1)=-\lambda x_0(1)=\lambda.
            \end{align*}
            Además, es claro que \(\norm{f_i}_{X^*}=1\), \(i=1,2\) ya que para todo \(a\in[0,1]\), \(\abs{x(a)}\leq \norm{x}_\infty\Rightarrow \norm{f_i}_{X^*}\leq 1\) y las funciones constantes \(x_1(t)=1\), \(x_2(t)=-1\) realizan el supremo para \(f_1,f_2\), respectivamente. 
        \end{itemize}
    \end{enumerate}
\end{Solucion}

\newpage
% Problema 8
\begin{Problema}
  Suponga que \(X\) es un espacio de Banach.
  \begin{enumerate}[(a)]
    \item Pruebe que \(X\) es reflexivo si y solo si \(X^{\ast}\) es reflexivo.  
    \item Demuestre que si \(X^{\ast}\) es separable,  entonces \(X\) es separable. 
    (Sugerencia: Para cada \(n\in\N\) , escoja \(x_n\in X\) con \(\norm{x_n} = 1\) y 
    \(\abs{f_n(x_n)} \ge \frac{1}{2} \norm{f_n}\), donde \(f_n \in X^{\ast}\) es un subconjunto contable denso,
    y demuestre (por contradicción) que \(Gen_{\K_{c}}(\left\{ x_n \right\}_{n}) = X\), donde 
    \(Gen_{\K_c}(S)\) denota el conjunto de combinaciones lineales finitas de \(S\), con coeficientes en 
    \(\K_{c} = \Q\) cuando \(\K=\R\) y \(\K_c = \Q + i \Q\) cuando \(\K = \C\). 
    Note que la combinación de esta proposición y la Pregunta 2 muestra que, en general, 
    \((L^{\infty})^{\ast} \not\cong L^1\).
  \end{enumerate}
\end{Problema}
\begin{Solucion}
  Denotemos \(X^{(0)} \coloneqq X\) y \(X^{(i)} \coloneqq (X^{(i-1)})^{\ast}\) para \(i>0\).   
  \begin{enumerate}[(a)]
    \item 
    \framebox{\(\implies\)}: Supongamos que \(X^{(0)}\) es reflexivo. Entonces, el mapa
    \begin{alignat*}{1}
      J^{(0)} \colon X^{(0)} &\to X^{(2)}\\
      x &\mapsto 
      \begin{aligned}
        ev_{x} \colon X^{(1)} &\to \K\\
        x^{(1)} &\mapsto x^{(1)}(x)
      \end{aligned}.
    \end{alignat*}
  es un isomorfismo isométrico. Queremos probar lo mismo para el mapa:
  \begin{alignat*}{1}
    J^{(1)} \colon X^{(1)} &\to X^{(3)}\\
    x^{(1)} &\mapsto
    \begin{aligned}
      ev_{x^{(1)}} \colon X^{(2)} &\to \K\\
      x^{(2)} &\mapsto x^{(2)}(x^{(1)}) 
    \end{aligned}.
  \end{alignat*}
  Para esto, basta probar que es sobreyectivo. 

  Dado que \(J^{(0)}\) es sobreyectivo, todo elemento \(x_0^{(2)} \in X^{(2)}\) se
  puede escribir como \(J^{(0)}(x_0)\) para algún \(x_0 \in X\).
  Sea \(x^{(3)} \in X^{(3)}\) se tiene que 
  \begin{displaymath}
    x^{(3)}(x_0^{(2)})
    = x^{(3)} ( J^{(0)}(x_0) )
    = (x^{(3)}\circ J^{(0)}) (x_0)
  \end{displaymath}
  Denotemos por \(x^{(1)} \coloneqq x^{(3)} \circ J^{(0)} \in X^{(1)}\). Luego,
  \begin{displaymath}
    x^{(3)}(x_0^{(2)}) = x^{(1)}(x_0) = J^{(0)}(x_0)(x^{(1)}) = x_0^{(2)}(x^{(1)}).
  \end{displaymath}
  Y el último término lo podemos expresar como \(J^{(1)}(x^{(1)})(x_0^{(2)})\).
  Esto demuestra la sobreyectividad.

  \framebox{\(\impliedby\)}: Supongamos que \(X^{(1)}\) es reflexivo. Definamos \(J^{(0)}, J^{(1)}\) de la misma manera. Supongamos por contradicción que \(J^{(0)}(X)\neq X^{(2)}\). Por Hahn-Banach, existe \(f\in X^{(3)}\) no nulo con \(f(x)\equiv 0\) para todo \(x\in J^{(0)}(X)\). Como \(X^{(1)}\) es reflexivo, el mapa \(J^{(1)}\) es un isomorfismo isometrico, por lo que existe \(g\in X^{(1)}\) con \(f=J^{(1)}(g)\). Luego, para todo \(X\in X^{(0)}\)
  \begin{align*}
      0&=f(J^{(0)}(x))\\
      &=J^{(1)}(g)\br{J^{(0)}(x)}\\
      &=J^{(0)}(x)(g)\\
      &=g(x).
  \end{align*}
  Luego, \(g\equiv 0\Rightarrow f\equiv 0\), lo que es una contradicción.
  \item
  Supongamos \(X^{\ast}\) separable. Consideremos la esfera unitaria en \(X^{\ast}\) 
  \begin{displaymath}
    S^{(1)} \coloneqq \left\{ \norm{x^{(1)}} = 1 \right\}.
  \end{displaymath}
  Luego, \(S^{(1)}\) también es separable. Sean \(\left\{ x^{(1)}_n \right\}_{n\in\N}\)
  elementos densos y numerables. Asociemos, para cada \(n\), un elemento
  \(x_n \in X\) tal que \(0 < \delta \le \abs{x^{(1)}_n (x_n)} \le 1\). Probaremos que 
  esos \(x_n\) son densos (en el sentido del espacio generado) en \(X\). 
  Buscando una contradicción, supongamos que no lo son. Entonces existe un \(x^{(1)}\in S^{(1)}\)
  tal que \(x^{(1)}\) se anula en todo \(U = \Gen_{\K_{c}}(\left\{ x_n \right\} )\) pero no es nulo
  fuera de \(U\). 
  (equivalentemente, existe un abierto en \(X\setminus U\) y podemos definir un funcional que se
  anule en todo menos ese abierto. Reescalando lo podemos pedir unitario).
  Por densidad en \(S^{(1)}\), existe \(N\) tal que \(\norm{x^{(1)}-x_N^{(1)}} > \delta/2\).
  Luego,
  \begin{displaymath}
    \delta \le \abs{x_N^{(1)}(x_N)} = \norm{x_N^{(1)}(x_N) - x^{(1)}(x_N)}
    \le \delta/2.
  \end{displaymath}
  Dada la contradicción, concluimos que \(x^{(1)}\) es nulo en \(X\) y por lo tanto
  los \(\Gen_{\K_c}(\left\{ x_n \right\})\) es denso y numerable en \(X\).

  La numerabilidad viene de la numerabilidad de los coeficientes y las sumas finitas del generado.
  La densidad viene de la densidad de los coeficientes y lo que acabamos de probar.
\end{enumerate}
\end{Solucion}

\end{document}

