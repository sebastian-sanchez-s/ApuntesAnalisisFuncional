\documentclass[11pt]{article}

% Page set up
\usepackage[margin=3cm]{geometry}

% Document font and symbols
\usepackage{mathptmx} % Times New Roman
\usepackage{amsmath,amsfonts,amsthm,amssymb}
\usepackage{esint}

% Language formatting
\usepackage[utf8]{inputenc}
\usepackage[T1]{fontenc}
\usepackage[spanish, es-noshorthands]{babel}

% Graph and Drawings
\usepackage{tikz,tikz-cd,float}
\usepackage{wrapfig}
\usetikzlibrary{babel}
\usepackage{pgfplots}
\usepackage[framemethod=default]{mdframed}
\usepackage{framed}

% Tools
\usepackage{multicol}
\usepackage{array}
\usepackage{hyperref}
\usepackage{xcolor}
\usepackage{etoolbox,mathtools}
\usepackage[shortlabels]{enumitem}

% Mdframed template
\mdfsetup{innertopmargin=5pt,%
	frametitlealignment=\raggedright,%
	frametitlefont=\texttt,%
	linewidth=2pt,%
	topline=false,rightline=false, bottomline=false,%
	frametitleaboveskip=\dimexpr-1\ht\strutbox\relax,}

% Problem Env
\newcounter{ProblemCounter}
\newenvironment{Problema}[1][]{%
	\vspace{1em}
	\refstepcounter{ProblemCounter}
  \noindent\tikz{\draw (0,0)--(\linewidth, 0) node[midway,fill=white]{{{\texttt{PROBLEMA}}~\theProblemCounter}};}
  %\noindent\texttt{PROBLEMA}~\theProblemCounter~
	}{
    \ \newline
    \tikz{\draw (0,0) -- (\linewidth, 0);}
    \vspace{1em}
	}

% Solution Env
\newenvironment{Solucion}[1][]
{%
  \newline
	\noindent{\ttfamily SOLUCIÓN}~
}%
{%
	%\hfill\(\blacksquare\)
}

% Resize abs and norm
\DeclarePairedDelimiter{\abs}{\lvert}{\rvert}
\DeclarePairedDelimiter{\norm}{\|}{\|}
\makeatletter
\let\oldabs\abs
\def\abs{\@ifstar{\oldabs}{\oldabs*}}
\let\oldnorm\norm
\def\norm{\@ifstar{\oldnorm}{\oldnorm*}}
\makeatother

% Shortcuts
\newcommand{\K}{\mathbb{K}}
\newcommand{\R}{\mathbb{R}}
\newcommand{\N}{\mathbb{N}}
\newcommand{\Z}{\mathbb{Z}}
\newcommand{\T}{\mathbb{T}}

\DeclareMathOperator{\esssup}{esssup}

% Header
\newcommand{\header}[2]
{
\begin{minipage}{.15\textwidth}
	\includegraphics[width=\textwidth,height=\textheight,keepaspectratio]{LogoUC}
\end{minipage}
\hspace{1em}
\begin{minipage}{.75\textwidth}
	\vspace{2em}
	{\scshape
	Pontificia Universidad Católica de Chile\\
	Facultad de Matemáticas\\
	Docente: Nikola Kamburov\\
	Ayudante: Matías Díaz
	}
\end{minipage}

\medskip

\begin{center}
	{\textbf{MAT2555 {-} Análisis Funcional}} \\[1ex]
	{{\large Tarea #1 {-} Omar Neyra, Sebastián Sánchez}}
\end{center}

\medskip
}

\begin{document}

\header{2}{}

% Problema 1
\begin{Problema}
  Sea 
  \((H, \langle \cdot,\cdot \rangle)\) un espacio de Hilbert y 
  \(T \in B(H,H)\) un operador lineal acotado.
  Demuestre que existe un único operador \(T^{\ast} \in B(H,H)\) tal que
  \begin{displaymath}
    \langle Tx,y \rangle
    =
    \langle x, T^{\ast} y \rangle
    \qquad
    \forall x,y\in H.
  \end{displaymath}
  El operador \(T^{\ast}\colon H\to H\) se llama el \textbf{operador adjunto} de \(T\).
  Pruebe que \(T^{\ast}\) satisface 
  \begin{enumerate}
    \item \(\norm{T^{\ast}} = \norm{T}\).
    \item \((T^{\ast})^{\ast} = T\). 
  \end{enumerate}
\end{Problema}

  Definamos el operador lineal \(x \mapsto \langle Tx, y \rangle\) para \(y\) fijo. 
  Luego,
  \begin{displaymath}
    \abs{\langle Tx, y \rangle} \le \norm{Tx} \norm{y}
    \implies \norm{\langle T\cdot, y \rangle} < \infty.
  \end{displaymath}
  Así, por el teorema de representación de Riesz se tiene que existe un único \(y^{\ast}\) 
  tal que
  \begin{displaymath}
    \langle Tx, y \rangle = \langle x, y^{\ast} \rangle.
  \end{displaymath}
  Definamos \(T^{\ast} y = y^{\ast}\). Por Riesz está bien definido, además, \(T^{\ast}\)
  es lineal y acotado. En efecto, consideremos \(T^{\ast} 0\), luego,
  \begin{displaymath}
    \langle x, T^{\ast} 0 \rangle = \langle Tx, 0 \rangle = 0
    \quad\forall x\in X
    \quad\therefore\quad T^{\ast} 0 = 0.
  \end{displaymath}
  Por otro lado, si tomamos \(\lambda y_1 + y_2\) tenemos que
  \begin{align*}
    \langle x, T^{\ast}( \lambda y_1 + y_2 ) \rangle
    &=
    \langle Tx, \lambda y_1 + y_2 \rangle
    \\&=
    \langle Tx, \lambda y_1 \rangle
    +
    \langle Tx, y_2 \rangle
    \\&=
    \overline{\lambda} \langle Tx, y_1 \rangle
    +
    \langle Tx, y_2 \rangle
    \\&=
    \overline{\lambda} \langle x, T^{\ast} y_1 \rangle
    +
    \langle x, T^{\ast} y_2 \rangle
    \\&=
    \langle x, \lambda T^{\ast}y_1 + T^{\ast}y_2 ) \rangle.
  \end{align*}
  para todo \(x\in X\) y por lo tanto \(T^{\ast}(\lambda y_1 + y_2) = \lambda T^{\ast}y_1 + T^{\ast}y_2\).  

  Ahora veamos que es acotado. Sea \(y\) unitario. Luego, 
  \begin{displaymath}
    \norm{T^{\ast} y}^{2}
    =
    \abs{\langle T^{\ast} y, T^{\ast} y \rangle}
    =
    \abs{\langle T(T^{\ast} y), y \rangle}
    \le
    \norm{T(T^{\ast} y)} \norm{y}
    \le
    \norm{T} \norm{T^{\ast} y}
  \end{displaymath}
  Así que \(\norm{T^{\ast} y} \le \norm{T} \) y por lo tanto \(\norm{T^{\ast}} \le \norm{T} <\infty\).  

  Para ver la unicidad, supogamos que existe otro operador lineal acotado \(S\) 
  que satisface lo mismo. Luego,
  \begin{displaymath}
    \langle x, T^{\ast} y\rangle
    =
    \langle x, S y \rangle
    \implies
    \langle x, T^{\ast}y - Sy \rangle = 0
    \quad\forall y\in Y, \forall x\in X.
  \end{displaymath}
  Por lo tanto \(T^{\ast} = S\). 

  Ahora veamos el resto.
  \begin{enumerate}
    \item Ya tenemos una igualdad. Queda ver el converso, es decir, que \(\norm{T} \le \norm{T^{\ast}}\).
    Sea \(x\) de norma \(1\). Luego,  
    \begin{alignat*}{1}
      \norm{T x}^{2}
      \\&=
      \abs{\langle Tx, Tx \rangle}
      \\&=
      \abs{\langle x, T^{\ast} (Tx) \rangle}
      \\&\le
      \norm{T^{\ast}(Tx)} \norm{x}
      \\&\le
      \norm{T^{\ast}} \norm{Tx}
      \Rightarrow \norm{Tx} \le \norm{T^{\ast}}.
    \end{alignat*}
    Tomando supremo se obtiene lo pedido.

    \item Sean \(x,y \in X\times Y\) cualquiera, luego:
    \begin{displaymath}
      \langle T^{\ast}x, y \rangle
      =
      \overline{ \langle y, T^{\ast}x \rangle }
      =
      \overline{ \langle Ty, x \rangle }
      =
      \langle x, Ty \rangle.
    \end{displaymath}
    Por lo tanto \(T = (T^{\ast})^{\ast}\). 
  \end{enumerate}

% Problema 2
\begin{Problema}
  Sea \((H, \langle \cdot, \cdot \rangle)\) un espacio de Hilbert separable y 
  \(\lbrace x_n \rbrace_{n\in \N} \subset H\) una sucesión acotada: \(\sup_{n} \norm{x_n} < \infty\).
  Demuestre que existe un \(x\in H\) y una subsucesión \(\lbrace x_{n_k} \rbrace_{k}\) tal que
  \begin{displaymath}
    \lim_{k\to\infty} \langle x_{n_k}, y \rangle
    =
    \langle x,y \rangle
    \quad
    \forall y\in H.
  \end{displaymath}
  Se dice que \(\lbrace x_{n_k} \rbrace\) \textbf{converge débilmente} a \(x\).

  (\textit{Sugerencia:} tome \(y\) igual a cada uno de los miembros de una base
  ortonormal y ejecute un argumento diagonal).

\end{Problema}
  Como \(H\) es un espacio de Hilbert separable, existe una base ortonormal 
  \(\lbrace e_n \rbrace_{n\in \N}\).

  Consideremos la sucesión \(S_1 = \lbrace \langle x_n, e_1 \rangle \rbrace \subset \K\). 
  Dado que la sucesión \(\left\{ x_n \right\}\) es acotada, la sucesión \(S_1\) es acotada
  (aplicando Cauchy-Schwarz). Luego, por Heine{-}Borel existe una subsucesión convergente.
  Denotemos a los elementos de la primera entrada de esa sucesión por \(\lbrace x_{n,1} \rbrace\).
  i.e.
  \begin{displaymath}
    \lbrace \langle x_{n,1}, e_1 \rangle \rbrace_{n} \text{ converge en } \K.
  \end{displaymath}
  
  Consideremos ahora la sucesión \(S_2 = \lbrace \langle x_{n,1}, e_2 \rangle \rbrace\). 
  Por el mismo argumento anterior existe una subsucesión convergente. 
  Denotemos a los elementos de \(\lbrace x_{n,1} \rbrace\) en \(S_1\) que aparecen en esta 
  subsucesión 
  por \(\lbrace x_{n,2} \rbrace\).
  Entonces tenemos que 
  \begin{displaymath}
    \lbrace \langle x_{n,2}, e_2 \rangle \rbrace_{n} \text{ converge en } \K.
  \end{displaymath}
  
  Inductivamente tendremos sucesiones \(\lbrace x_{n,k} \rbrace\) tales que
  \begin{displaymath}
    \lbrace \langle x_{n,k}, e_k \rangle \rbrace_{n} \text{ converge en } \K.
  \end{displaymath}
  Digamos que \(\langle x_{n,k}, e_k \rangle \xrightarrow{n\to\infty} \lambda_k\in \K\).
  Nótese que
  \begin{displaymath}
    \langle x_{n,n}, e_k \rangle \xrightarrow{n\to\infty} \lambda_k.
  \end{displaymath}
  pues \(\lbrace x_{n,k+1} \rbrace_{n} \subset \lbrace x_{n,k} \rbrace_{n}\).

  Definamos \(x = \sum_{k} \lambda_k e_k \). Afirmamos que \(x\) existe y que 
  \(\lim_{k\to \infty} \langle x_{n,n}, y \rangle = \langle x, y \rangle\) para todo 
  \(y\in Y\). 

  \underline{Existencia:} Usaremos que
  \begin{displaymath}
    \sum_{k} c_k e_k \in H
    \iff
    \lbrace c_k \rbrace_k \in \ell^{2}.
  \end{displaymath}
  
  Veamos entonces que los coeficientes están en \(\ell^{2}\). Consideremos las
  sumas parciales:
  \begin{displaymath}
    \sum_{k \le N} \abs{\lambda_k}^{2}
    =
    \sum_{k \le N} \abs{\lim_{n\to\infty} \langle x_{n,n}, e_k \rangle}^{2}
    =
    \lim_{n\to\infty} \sum_{k\le N} \abs{\langle x_{n,n}, e_k \rangle}^{2}
    \le
    \limsup_{n} \norm{x_{n,n}}^{2}
    \le
    \sup_{n} \norm{x_n}^{2}.
  \end{displaymath}
  Luego, \(x\) existe.

  \underline{Propiedad:}
  Sea \(y\in H\). Luego,
  \begin{displaymath}
    \abs{ \langle x_{n,n}, v \rangle - \langle x, v \rangle }
    \le
    \underbrace{
    \abs{ 
      \langle x_{n,n}, \sum_{k\le N} \lambda_{k} e_k \rangle 
      -
      \langle x, \sum_{k\le N} \lambda_{k} e_k \rangle
    }}_{\to 0}
    +
    2 
    \, \underbrace{\sup_{n} x_n}_{\text{acotado}} 
    \, \underbrace{\norm{v - \sum_{k\le N} \lambda_{k} e_k}}_{\to 0}.
  \end{displaymath}
  Por lo tanto se concluye el resultado.

% Problema 3
\begin{Problema}[Sumabilidad de Abel de la serie de Fourier]
  Sea \(f \in C(\mathbb{T})\) y sea 
  \begin{displaymath}
    f(\theta) \sim \sum_{n\in \Z} \hat f(n) e_{n}(\theta)
  \end{displaymath}
  la serie de Fourier de \(f\).
  \begin{enumerate}[(a)]
    \item Demuestre que para todo \(r\in [0,1)\) la serie 
    \begin{displaymath}
      u(r,\theta) \coloneqq \sum_{n\in \Z} r^{\abs{n}} \hat f(n) e_{n}(\theta)
    \end{displaymath}
    define una función \(u(r,\theta) \in C^2([0,1]\times \mathbb{T})\), la cual satisface 
    la ecuación de Laplace en coordenadas polares \((r,\theta)\):
    \begin{displaymath}
      \Delta u 
      =
      \left( 
        \partial_r^2 + \frac{1}{r} \partial_r + \frac{1}{r^2} \partial_{\theta}^2
      \right) u
      = 0
    \end{displaymath}
    en el disco unitario \(\mathbb{D} = \lbrace 0\le r < 1, \theta \in \mathbb{T} \rbrace\). 
    
    \item Pruebe que podemos expresar \(u(r,\theta) = P_r \ast f(\theta)\) como una convolución
    de \(f\) con el kernel de Poisson
    \begin{displaymath}
      P_r(\theta) \coloneqq
      \frac{1}{2\pi}
      \sum_{n\in\Z} r^{\abs{n}} e^{in\theta}
      =
      \frac{1}{2\pi}
      \frac{1-r^{2}}{1-2r\cos(\theta) + r^2}.
    \end{displaymath}
    
    \item Demuestre que
    \begin{displaymath}
      \lim_{r\to 1} u(r,\theta) = f(\theta)
    \end{displaymath}
    uniformemente en \(\theta\). (\textit{Sugerencia:} Verifique que \(\lbrace P_r \rbrace_{r\in[0,1)}\)
    es una familia de buenos kernels en \(L^1(\mathbb{T})\)). 
  \end{enumerate}
\end{Problema}
\begin{enumerate}[(a)]
  \item Notar que \(f\) y \(e_n\) son acotadas (son continuas en un compacto). Luego,
  \begin{displaymath}
    \abs{u(r,\theta)}
    \le
    C \sum_{n\in \Z} r^{\abs{n}}
    \le
    2 C \sum_{n \ge 0} r^{n}
    < \infty
  \end{displaymath}
  para alguna constante \(C>0\). 
  Así, la serie es absoluta y uniformemente convergente. 
  Luego, podemos derivar la serie término a término. 
  De esta forma se tiene que
  \begin{alignat*}{1}
    \partial_{r} u(r,\theta) &= \sum_{n\in\Z} \abs{n} r^{\abs{n}-1} \hat f(n) e_{n}(\theta)
    \\
    \partial_{\theta} u(r,\theta) &= \sum_{n\in\Z} i n e_n(\theta) r^{\abs{n}} \hat f(n)
  \end{alignat*}

  Dado que (para \(r \ne 0\)) 
  \begin{alignat*}{1}
    \sum_{n\in\Z} \abs{n} r^{\abs{n}-1} \hat f(n) e_{n}(\theta)
    &\le
    \frac{2C}{r} \sum_{n\ge 0} n r^n
    \\&\le
    \frac{2C}{r} \lim_{N\to\infty} \frac{r - (N+1)r^{N+1} + Nr^{N+2}}{(1-r)^2}
    < \infty
  \end{alignat*}

  La serie derivada también es absoluta y uniformemenente convergente, así que podemos volver a
  derivar término a término (el argumento para la serie derivada con respecto a \(\theta\) es análogo).
  \begin{alignat*}{1}
    \partial_{r}^2 u(r,\theta) &= \sum_{n\in\Z} \abs{n} (\abs{n}-1) r^{\abs{n}-2} \hat f(n) e_n(\theta)
    \\
    \partial_{\theta}^2 u(r,\theta) &= \sum_{n\in\Z} -n^2 e_n(\theta) r^{\abs{n}} \hat f(n)
  \end{alignat*}
  Esta serie también es absoluta y uniformemente convergente, por lo tanto, la función \(u\in C^2\). 

  Aplicando la fórmula vemos que
  \begin{align*}
    \Delta u 
    &= \abs{n}(\abs{n}-1) r^{\abs{n}-2} \cdot \star + \abs{n} r^{\abs{n}-2} \cdot\star - \frac{n^2}{r^2} \cdot \star
    \\&= (\abs{n}^2 r^{\abs{n}-2} - n^2 r^{\abs{n}-2}) \cdot \star
    \\&= 0
  \end{align*}
  donde \(\star = \hat f(n) e_n(\theta)\).

  \item Expandamos los términos:
  \begin{displaymath}
    u(r,\theta)
    =
    \sum_{n\in\Z} \frac{1}{2\pi} \int_{0}^{2\pi} r^{\abs{n}} f(x) e^{in(\theta-x)} \, dx
  \end{displaymath}
  Por TCD podemos cambiar la integral con la serie:
  \begin{displaymath}
    u(r,\theta)
    =
    \int_{0}^{2\pi} f(x) \sum_{n\in\Z} \frac{1}{2\pi} \cdot r^{\abs{n}} e^{in(\theta-x)} \, dx
    =
    (\sum_{n\in\Z} r^{\abs{n}} e^{in\theta}) \ast f (r,\theta).
  \end{displaymath}

  \item Supongamos momentáneamente que la familia \(\lbrace P_r \rbrace_{r\in[0,1)}\) es de buenos
  kernels. En tal caso tenemos que
  \begin{align*}
    \abs{f(\theta) - u(r,\theta)}
    &=
    \abs{
      \frac{1}{2\pi} 
      \int_{-\pi}^{\pi} f(\theta) \, dx
      -
      \frac{1}{2\pi} 
      \int_{-\pi}^{\pi} f(\theta-x) r^{\abs{n}} e^{-inx} \, dx
    }
    \\&\le
    \frac{1}{2\pi} 
    \int_{-\pi}^{\pi} 
      \abs{f(\theta) - f(\theta-x)} 
      r^{\abs{n}} e^{-inx} 
    \, dx
  \end{align*}
  De esta forma, tomando una bola alrededor del origen podemos controlar el término
  con \(f\) por continuidad y el resto lo podemos achicar porque la cola de los kernel
  se va a cero. Concretamente: dado que \(f\) es continua, para todo \(\epsilon > 0\) existe
  un \(\delta > 0\) tal que 
  \begin{displaymath}
    \abs{x} < \delta \implies \abs{f(\theta) - f(\theta+x)} < \epsilon.
  \end{displaymath}
  Por otro lado, para el mismo \(\delta\) se satisface que
  \begin{displaymath}
    \int_{\delta \le \abs{x} \le \pi} \abs{P_r(x)}\, dx \xrightarrow{r\to 1} 0.
  \end{displaymath}
  De esta forma tenemos que
  \begin{displaymath}
    \abs{f(\theta) - u(r,\theta)}
    \le
    \epsilon \int_{\abs{x} \le \delta} P_{r}(x) \, dx
    +
    \sup_{\mathbb{T}} \abs{f}
    \int_{\delta \le \abs{x} \le \pi} P_r(x) \, dx.
  \end{displaymath}
  Tomando \(r\) suficientemente cercado a \(1\) podemos hacer el término no
  constante del segundo sumando menor a \(\epsilon\). Notando el que término
  que acompaña a \(\epsilon\) en el primer sumando está acotado (porque son buenos kernels)
  concluimos que:
  \begin{displaymath}
    \abs{f(\theta) - u(r,\theta)}
    \le
    \epsilon C.
  \end{displaymath}

  \underline{Demostración que \(\lbrace P_r \rbrace_{r\in[0,1)}\) es una familia de buenos kernels:}
  \begin{enumerate}
    \item Integra \(1\): Por la convergencia absolutamente uniforme podemos cambiar la integral con 
    la serie, resultando en
    \begin{displaymath}
      \frac{1}{2\pi}
      \int_{\T} \sum_{n\in\Z} r^{\abs{n}} e^{in\theta} \, dt
      =
      \frac{1}{2\pi}
      \sum_{n\in\Z} r^{\abs{n}} \underbrace{\int e^{in\theta} \, dt}_{=0\text{ si } n\ne 0}
      =
      1.
    \end{displaymath}

    \item Cota absolutamente uniforme: Dado que \(P_{r}(t)\) es positiva,  
    \begin{displaymath}
      \int_{\T} \abs{P_r(t)} \, dt
      =
      \int_{\T} P_r(t) \, dt = 1.
    \end{displaymath}
    para cualquier \(r\). 

    \item Colas se van a cero: Sea \(0 < \delta < \pi/2\). Luego, existe \(r\in(0,1)\) tal que
    \(\delta = \arccos(r/2)\). Por la identidad dada se sigue que
    \begin{displaymath}
      \int_{\delta \le \abs{x} \le \pi} P_r(x)
      \le
      \frac{1}{2\pi}
      \int_{\delta \le \abs{x} \le \pi} \frac{1-r^2}{1+r^2-2r\cos(x)}
      \le
      C (1-r^2).
    \end{displaymath}
    Similarmente, si \(\delta > \pi/2\) podemos acotar directamente 
    \begin{displaymath}
      \int_{\delta \le \abs{x} \le \pi} P_r(x)
      \le
      \frac{1}{2\pi}
      \int_{\pi/2 \le \abs{x} \le \pi} \frac{1-r^2}{1+r^2-2r\cos(x)}
      \le
      C (1-r^2).
    \end{displaymath}
    En ambos casos, tomando \(r\to 1\) nos da lo pedido.  
  \end{enumerate}
\end{enumerate}

% Problema 4
\begin{Problema}
  Denote por \(F_N\in L^1(\mathbb{T})\), \(N\in\N\), el kernel de Fejèr. Sea \(\sigma_N f \coloneqq f\ast F_N\).
  Demuestre que para todo \(p\in [1,\infty]\)
  \begin{displaymath}
    \norm{ \sigma_N f }_{L^p(\mathbb{T})}
    \le
    \norm{f}_{L^p(\mathbb{T})}
    \quad\forall f\in L^{p}(\mathbb{T}).
  \end{displaymath}
  
  \textit{Sugerencia:} Utilice la desigualdad de Hölder en estimar
  \begin{displaymath}
    \int_{\mathbb T} F_N(y) \, \abs{f(x-y)} \, dy
    =
    \int_{\mathbb T} F_N(y)^{1/q} F_N(y)^{1/p} \, \abs{f(x-y)}\, dy
    \qquad
    \frac 1 p + \frac 1 q = 1
  \end{displaymath}
  en combinación con el \textit{Teorema de Fubini-Tonelli}:
  \begin{displaymath}
    \int_{\mathbb T}
    \left( 
      \int_{\mathbb T}
        K(x,y)\, dx
    \right) \, dy
    =
    \int_{\mathbb T}
    \left( 
      \int_{\mathbb T}
        K(x,y)\, dy
    \right) \, dx
  \end{displaymath}
  para toda \(K\colon \mathbb{T}^2 \to [0,\infty]\) Lebesgue medible. 
\end{Problema}
\underline{Caso \(1 < p < \infty\)}: Equivalentemente, debemos probar que
\begin{displaymath}
  \norm{\sigma_N f}_{p}^{p} \le \norm{f}_{L^p}^p
\end{displaymath}
Expandamos el término de la izquierda
\begin{align*}
  \norm{\sigma_N f}_{L^p}^{p}
  &=
  \int_{\mathbb T}
    \abs{
      f\ast F_N (x) 
    }^p
  \, dx
  \\&\le
  \int_{\mathbb T}
    \left(
      \int_{\mathbb T}
        \abs{f(x-y)} F_N(y)
      \, dy
    \right)^p
  \, dx
  =
  \int_{\mathbb T}
    \left(
      \int_{\mathbb T}
        \abs{f(x-y)} F_N(y)^{1/p} F_N(y)^{1/q}
      \, dy
    \right)^p
  \, dx
\end{align*}
con \(1/p + 1/q = 1\). Luego, por la desigualdad de Hölder  
\begin{displaymath}
  \norm{\sigma_N f}_{L^p}^{p}
  \le
  \norm{F_N}_{L^1}^{p/q}
  \int_{\mathbb T}
    \int_{\mathbb T}
      \abs{f(x-y)}^p
      F_N(y)
    \, dy
  \, dx
\end{displaymath}
El integrando es medible (\(f\) es medible y \(F_N\) es suave) así que podemos aplicar Tonelli{-}Fubini:
\begin{displaymath}
  \norm{\sigma_N f}_{L^p}^{p}
  \le
  1\cdot
  \int_{\mathbb T}
    F_N(y)
    \int_{\mathbb T}
      \abs{f(x-y)}^p
    \, dx
  \, dy
\end{displaymath}
Como estamos integrando sobre el círculo y la función es periódica, tenemos que  
\begin{displaymath}
  \int_{\mathbb T} \abs{f(x-y)}^p \, dx
  =
  \int_{\mathbb T} \abs{f(x)}^p \, dx
  =
  \norm{f}_{L^p}^{p}
\end{displaymath}
Por lo tanto,
\begin{displaymath}
  \norm{\sigma_N f}_{L^p}^{p} \le \norm{f}^{p}_{p}.
\end{displaymath}

\underline{Caso \(p=1\)}: Ya apareció en el cálculo anterior, pero aquí va:
\begin{displaymath}
  \norm{\sigma_N f}_{L^1}
  \le 
  \int_{\T} \int_{\T} \abs{f(x-y)} F_N(y)\, dy\, dx.
\end{displaymath}
Por Tonelli{-}Fubuni y la observación anterior de integrar sobre el círculo tenemos que
\begin{displaymath}
  \norm{\sigma_N f}_{L^1}
  \le
  \int_{\T} \int_{\T} \abs{f(x-y)} F_N(y)\, dx\, dy
  \le
  \norm{f}_{L^1} \underbrace{\norm{F_N}_{L^1}}_{=1}.
\end{displaymath}

\underline{Caso \(p=\infty\)}:
\begin{displaymath}
  \norm{\sigma_N f}_{L^\infty}
  \le 
  \esssup_{\T} \int_{\T} \abs{f(x-y)} F_N(y)\, dy.
\end{displaymath}

% Problema 5
\begin{Problema}
  Sea \(f\) una función \(2\pi\){-}periódica en \(\R\) de clase \(C^1\).
  \begin{enumerate}[(a)]
    \item Pruebe que \(\hat f'(n) = in \hat f(n)\). Concluya que \(\lim_{\abs{n}\to\infty} \abs{n}\hat f(n) = 0\).  
    \item Suponga que adicionalmente \(\int_{\T} f(x) dx = 0\). Utilizando la identidad de Parseval,
    demuestre la \textit{desigualdad de Wirtinger}:
    \begin{displaymath}
      \int_{-\pi}^{\pi} \abs{f(x)}^{2} \,dx
      \le
      \int_{-\pi}^{\pi} \abs{f'(x)}^{2} \,dx
    \end{displaymath}
  \end{enumerate}
\end{Problema}
\begin{enumerate}[(a)]
  \item Haciendo integración por partes tenemos
  \begin{displaymath}
    \hat f'(n) 
    = \frac{1}{2\pi} \int_{-\pi}^{\pi} f'(x) e^{-inx} \, dx
    = \frac{1}{2\pi} \left[ f(x) e^{-inx} \right]_{-\pi}^{\pi} + in \int_{T} f(x) e^{-inx} \, dx 
    = in \hat f(n).
  \end{displaymath}
  Como \(f'\) es continua, en particular es \(L^1\) integrable. Aplicando Riemann-Lebesgue tenemos:
  \begin{displaymath}
    \lim_{\abs{n}\to\infty} 
      \abs{n} \hat f(n)
    =
    \lim_{\abs{n}\to\infty} 
      \hat f'(n)
    =
    0.
  \end{displaymath}

  \item Por Parseval,
  \begin{alignat*}{2}
    \norm{f}_{L^2} 
    &= \sum_{n\in\Z} \abs{\hat f(n)}
    &&= \abs{\hat f'(0)} + \sum_{n\in\Z\setminus 0} \frac{\abs{\hat f'(n)}}{n}
    \\&\le 
    \sum_{n\in\Z} \abs{\hat f'(n)}
    &&= \norm{f'}_{L^2}
  \end{alignat*}
\end{enumerate}


% Problema 6
\begin{Problema}
  Considere la función en \(\T\), dada por
  \begin{displaymath}
    f(x) =
    \begin{cases}
      0 &, \abs{x} > \delta\\
      1-\abs{x}/\delta &, \abs{x} \le \delta
    \end{cases}
    , \delta \in (0,\pi).
  \end{displaymath}
  Pruebe que 
  \begin{displaymath}
    f(x) 
    = 
    \frac{\delta}{2\pi} 
    + 
    2 \sum_{n=1}^{\infty} 
      \frac{ 1-\cos n\delta }{ n^2 \pi \delta } \cos(nx).
  \end{displaymath}
\end{Problema}
Dado que la función está en \(L^2\), se tiene que \(S_N f \to f\). Para \(n\ne 0\) se tiene
que:
\begin{displaymath}
  \hat f(n)
  = 
  \frac{1}{2\pi}
  \int_{\abs{t} \le \delta} f(t) e^{-int}
  =
  \frac{1}{2\pi}
  \left( 
    \underbrace{ \int_{\abs{t} \le \delta} e^{-int} }_{I}
    -
    \frac{1}{\delta}
    \underbrace{ \int_{\abs{t} \le \delta} \abs{t} e^{-int} }_{J}
  \right) 
\end{displaymath}

Computemos \(J\): reordenando la integral y haciendo integración por partes tenenemos 
\begin{align*}
  J 
  &= \int_{-\delta}^{0} -te^{-int} + \int_{0}^{\delta} t e^{-int}
  \\&= \int_{0}^{\delta} te^{int} + \int_{0}^{\delta} t e^{-int}
  \\&= \int_{0}^{\delta} t (e^{int} + e^{-int})
  \\&= 2 \int_{0}^{\delta} t \cos(nt)
  \\&= 2 \left( 
    \left[ \frac{t\sin(nt)}{n} \right]_0^{\delta} 
    - \frac{1}{n} \int_{0}^{\delta} \sin(nt)
  \right) 
  \\&= 2 \left( 
    \frac{\delta \sin(n\delta)}{n}
    +
    \frac{\cos(n\delta)-1}{n^2}
  \right) 
\end{align*}

Ahora veamos \(I\):
\begin{displaymath}
  I 
  = \int_{-\delta}^{\delta} e^{-int}
  = \frac{ e^{in\delta} - e^{-in\delta} }{ in }
  = \frac{ 2\sin(n\delta) } { n }.
\end{displaymath}

Juntando todo:
\begin{align*}
  \hat f(n)
  &=
  \frac{1}{2\pi}
  \left( 
    \frac{ 2\sin(n\delta) } { n }
    -
    \frac{2\sin(n\delta)}{n}
    -
    2 \frac{\cos(n\delta)-1}{\delta n^2}
  \right) 
  \\&=
  \frac{1-\cos(n\delta)}{\delta \pi n^2}.
\end{align*}

Luego, el coeficiente es par (\(\hat f(n) = \hat f(-n)\)), así que
\begin{displaymath}
  \sum_{n\in\Z\setminus 0} \hat f(n) e^{inx}
  =
  \sum_{n=1}^{\infty} \hat f(n) (e^{inx} + e^{-inx}) 
  =
  2 \sum_{n=1}^{\infty} \hat f(n) \cos(nx).
\end{displaymath}

Finalmente, para \(n=0\) tenemos que
\begin{displaymath}
  \hat f(0) 
  = \frac{1}{2\pi} \int_{\abs{t} \le \delta} (1 - \abs{t}/\delta)
  = \frac{1}{2\pi} (2\delta - \frac{2}{\delta} \frac{\delta^2}{2})
  = \frac{\delta}{2\pi}.
\end{displaymath}

Así, concluimos que
\begin{displaymath}
  f(x) = \frac{\delta}{2\pi} + 2\sum_{n=1}^{\infty} \frac{1-\cos(n\delta)}{n^2\delta n} \cos(nx).
\end{displaymath}



% Problema 7
\begin{Problema}
  Sea \(X\) un espacio normado y \(J\subset X\). Demuestre que 
  \(\lbrace f(x) \colon x \in J \rbrace\) es acotado para todo funcional lineal acotado
  \(f\in X'\) si y solo si \(J\) es acotado.
  \begin{displaymath}
    \exists M>0, \quad \norm{x} \le M \text{ para todo } x\in J.
  \end{displaymath}
\end{Problema}
\framebox{\(\Rightarrow\)}
Consideremos el funcional evaluación \(f_x\colon X' \to \K, f_x(x') = x'(x)\). 
Por Hahn{-}Banach este operador es una isometría.
Luego, \(\norm{x} = \norm{f_x(x')} < \infty\) para todo \(x\in X\). 
Aplicando Banach{-}Steinhaus sobre el conjunto \(J\) tenemos que 
\begin{displaymath}
  \sup_{x\in J} \norm{x} = \sup_{x\in J} \norm{f_x} < \infty.
\end{displaymath}

\framebox{\(\Leftarrow\)} Supongamos que \(J\) es acotado. Sea \(x' \in X'\) un funcional
lineal acotado. Luego, para \(x\in J\) se tiene que 
\(\norm{x'(x)} \le \norm{x'} \norm{x} < \infty\) porque el funcional
es acotado y \(\norm{x} \le \sup_{x\in J} \norm{x} < \infty\) por hipótesis.

% Problema 8
\begin{Problema}
  Sean \(X,Y\) dos espacios de Banach y \(\lbrace T_n \rbrace_n \subset B(X,Y)\) una sucesión
  de operadores lineales acotados. Suponga que para todo \(x\in X\) el límite
  \(Tx \coloneqq \lim_{n\to\infty} T_n x\) existe. Pruebe que si \(x_n \to x\), entonces
  \(T_n x_n \to Tx\) para todo \(x\in X\).  
\end{Problema}

Por un corolario de Banach{-}Steinhaus visto en clases, el operador \(T\) es lineal y acotado.
Veamos la convergencia:
\begin{displaymath}
  \norm{ Tx - T_n x_n}
  \le
  \norm{ Tx - T_n x} + \norm{ T_n x - T_n x}
\end{displaymath}
Por la convergencia puntual, existe \(N_0\) tal que \(\norm{ Tx - T_n x } < 1/N_0\)
para \(n>N_0\). Por otro lado, el operador \(T_n\) es lineal y acotado, así que
\(\norm{T_n y} \le \norm{T_n} \norm{y}\); además, como \(x_n \to x\), existe \(N_1\) tal que
\(\norm{ x - x_n } < 1/N_1\). Así, tomando \(N \ge \max(N_0, N_1)\) se tiene que
\begin{displaymath}
  \norm{ Tx - T_n x_n}
  \le
  \norm{ Tx - T_n x} + \norm{ T_n x - T_n x}
  \le
  \frac{1}{N} + \frac{1}{N} \norm{T_n}
  \qquad\forall n \ge N.
\end{displaymath}
También por Banach{-}Steinhaus, \(\sup_{n\in \N} \norm{T_n} < M\) para algún \(M < \infty\).
Así, tomando \(N\to \infty\) se tiene lo pedido. 
\begin{displaymath}
  \norm{ Tx - T_n x_n} \le \frac{1}{N} (M+1) \xrightarrow{N\to\infty} 0.
\end{displaymath}



\end{document}
