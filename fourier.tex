\documentclass{article}

\usepackage[spanish]{babel}
\usepackage{amsmath, amsthm, amssymb}
\usepackage{mathtools}
\usepackage{tikz}

\newtheorem{Teorema}{Teorema}
\newtheorem{Lema}{Lema}
\newtheorem{Proposicion}{Proposición}
\newtheorem{Corolario}{Corolario}

\theoremstyle{plain}
\newtheorem{Problema}{Problema}

\theoremstyle{definition}
\newtheorem{Definicion}{Definición}

\newcommand{\T}{\mathbb{T}}
\newcommand{\C}{\mathbb{C}}
\newcommand{\R}{\mathbb{R}}
\newcommand{\Z}{\mathbb{Z}}
\newcommand{\N}{\mathbb{N}}
\newcommand{\F}{\mathcal{F}}

\DeclareMathOperator{\esssup}{esssup}

\newcommand{\abs}[1]{\lvert #1 \rvert}
\newcommand{\norm}[1]{\lVert #1 \rVert}
\newcommand{\normL}[2]{\lVert #2 \rVert_{L^#1}}

\begin{document}

\noindent\tikz{
\draw (0,0)--(\linewidth, 0) 
node[midway,above,fill=white]{\Large Series de Fourier}
node[midway,below,fill=white]{\small y Banach-Steinhaus}; }
\newline

Trabajaremos con funciones en \(L^2\) donde 
\begin{displaymath}
  L^2 \coloneqq 
  L^{2}(\T)
  =
  \left\{ 
    f\colon \R \to \C
    \mid 
    f \text{ sea medible, \(2\pi\){-}periódica y }
    \int_{\T} \abs{f(x)}^2 \, dx < \infty
  \right\} 
\end{displaymath}
y \(\T = [-\pi,\pi]\) con extremos identificados .i.e. el círculo. 
con el producto interno de \(L^2\)
\begin{displaymath}
  \langle f,g \rangle
  =
  \int_{\T} f(x) \overline{g(x)} \, dx.
\end{displaymath}

\begin{Proposicion}[Conjunto ortornormal]
  Las familia \(\lbrace (2\pi)^{-1/2} e^{inx} \rbrace_{n\ge 0}\) es un conjunto ortonormal
  en \(L^2\).
\end{Proposicion}
\begin{proof}
  Sean \(m > n > 0\). Luego,
  \begin{align*}
    \langle (2\pi)^{-1/2} e^{inx}, (2\pi)^{-1/2} e^{imx} \rangle
    &=
    (2\pi)^{-1}
    \int_{-\pi}^{\pi} 
      e^{inx} e^{-imx}
      \, dx
    \\&=
    (2\pi)^{-1}
    \int_{-\pi}^{\pi} 
      e^{ix(n-m)}
      \, dx
    \\&=
    -(2\pi)^{-1} (n-m)^{-1}
    \left( e^{ix(m-n)} \right)\big\vert_{-\pi}^{\pi}
    \\&=
    -(2\pi)^{-1} (n-m)^{-1}
    (\pm 1 \mp 1)
    = 0.
  \end{align*} 
  Por otro lado, si \(n=0\):
  \begin{displaymath}
    (2\pi)^{-1/2} \int_{\T} e^{\pm inx} = 0.
  \end{displaymath}
\end{proof}

\begin{Definicion}[Serie de Fourier]
  Sea \(f\in L^2\), su serie de Fourier es
  \begin{displaymath}
    \sum_{n\in\Z} \hat f(n) e_{n}
  \end{displaymath}
  donde \(e_n(x) = (2\pi)^{-1/2} e^{inx}\) y 
  \begin{displaymath}
    \hat f(n) = (2\pi)^{-1/2} \int_{\T} f(x) e^{-inx} \, dx
  \end{displaymath}
  es el \(n\){-}ésimo coeficiente de Fourier. La serie truncada
  \begin{displaymath}
    S_N f(x) = \sum_{\abs{n} \le N} \hat f(n) e_{n}(x)
  \end{displaymath}
  es la suma parcial de Fourier.
\end{Definicion}

\begin{Teorema}\label{main:thm}
  Si \(f\in L^2\), \(S_N f \to f\) en norma \(L^2\) cuando \(N\to \infty\).  
\end{Teorema}

Equivalentemente, la colección \(\lbrace e_n \rbrace_{n\in\Z}\) es una base ortonormal de \(L^2\).
Como ya tenemos la ortonormalidad, queda ver la maximalidad. Suponiendo que no es maximal, 
debe existir una función \(f \ne 0\) tal que
\begin{displaymath}
  \int_{T} f(x) e^{-inx} \,dx = 0 \quad\forall n\in \Z.
\end{displaymath}
La idea es probar que \(f = 0\) c.t.p. No obstante, esto es más complejo de lo que parece. 
El propósito de esta sección es buscar una forma alternativa de probar el Teorema~\ref{main:thm}.

\begin{Proposicion}[Caracterización de \(S_N f\) y kernel de Dirichlet]
  Si \(f\in L^2\), entonces 
  \begin{displaymath}
    S_N f(x) = \int_{\T} D_N(x-t) f(t) \, dt = D_N\ast f(x)
  \end{displaymath}
  donde 
  \begin{displaymath}
    D_N(x) = \begin{cases}
      \frac{2N+1}{2\pi} &, x=0 \\
      \frac{ \sin\left( Nx + x/2 \right)}{ 2\pi \sin(x/2)} &, x\ne 0
    \end{cases}
  \end{displaymath}
  es el kernel de Dirichlet. 
\end{Proposicion}
\begin{proof}
  Expandamos la definición
  \begin{align*}
    S_N f(x)
    &=
    \sum_{\abs{n} \le N}
      \frac{1}{2\pi}
      \int_{\T} f(t) e^{-int} \, dt
      \, e^{inx}
    \\&=
    \sum_{\abs{n} \le N}
      \frac{1}{2\pi}
      \int_{\T} f(t) e^{-in(t-x)} \, dt
    \\&\le
    \int_{\T} 
      f(t) 
      \frac{1}{2\pi}
      \sum_{\abs{n} \le N}
      e^{-in(t-x)}
    \, dt.
  \end{align*}
  Así que basta probar que \(2\pi D_N(x) = \sum_{\abs{n}\le N} e^{inx}\).

  \underline{Caso \(x=0\):}
  \begin{displaymath}
    \sum_{\abs{n} \le N} e^{in\cdot 0}
    = \sum_{\abs{n} \le N} 1
    = 2N+1.
  \end{displaymath}

  \underline{Caso \(x\ge 0\):} Tomemos \(w = e^{ix}\). Luego,
  \begin{align*}
    \sum_{\abs{n} \le N} w^n
    &=
    \sum_{0 \le n \le N} w^n
    +
    \sum_{1 \le n \le N} w^{-n}
    \\&=
    \frac{1-w^{N+1}}{1-w}
    +
    \frac{w^{-1} - w^{-N-1}}{1-w^{-1}}
    \\&=
    \frac{1-w^{N+1}}{1-w}
    +
    \frac{w^{-N}-1}{1-w}
    \\&=
    \frac{w^{-N} - w^{N+1}}{1-w}
    \\&=
    \frac{w^{-N-1/2} - w^{N+1/2}}{w^{-1/2} - w^{1/2}}
    \\&=
    \frac{ \sin\left( (N+1/2)x \right) }{ \sin\left( x/2 \right)}.
  \end{align*}
\end{proof}

\begin{Proposicion}[Propiedades del kernel de Dirichlet]
  \(D_N(x)\) satisface que
  \begin{enumerate}
    \item es par;
    \item es \(2\pi\){-}periódica;
    \item es suave;
    \item integra \(1\) sobre el círculo,
    \begin{displaymath}
      \int_{\T} D_N(x) \, dx = 1;
    \end{displaymath}
    \item para todo \(\delta > 0\) se tiene que
    \begin{displaymath}
      \int_{\delta \le \abs{x} \le \pi} D_N(x) \to 0
      \quad \text{cuando}\quad N\to \infty.
    \end{displaymath}
  \end{enumerate}
\end{Proposicion}
\begin{proof}
\begin{enumerate}
  \item 
  \begin{align*}
    D_N(x) 
    &= \frac{1}{2\pi} \frac{\sin((N+1/2)x)}{\sin(x/2)} 
    \\&= \frac{1}{2\pi} \frac{-\sin(-(N+1/2)x)}{-\sin(-x/2)} 
    = D_N(-x).
  \end{align*}

  \item
  \begin{align*}
    2\pi D_N(x) 
    &= \frac{\sin((N+1/2)x)}{\sin(x/2)} 
    \\&= \frac{-\sin((N+1/2)(x+2\pi))}{-\sin((x+2\pi)/2)} 
    = 2\pi D_N(x+2\pi).
  \end{align*}

  \item es cociente de funciones suaves.

  \item Recordar que \(2\pi D_N(x) = \sum_{\abs{n}\le N} e^{inx}\). 
  \begin{align*}
    \int_{\T} \sum_{\abs{n} \le N} e^{inx}\, dx
    &=
    \sum_{\abs{n}\le N} \int_{\T} e^{inx} \, dx
    = 2\pi.
  \end{align*}
\end{enumerate}
\end{proof}

\begin{Definicion}[Media de Cesàro]
  \begin{displaymath}
    \sigma_N f 
    = 
    \frac{1}{N} \sum_{0\le n < N} S_{n} f.
  \end{displaymath}
\end{Definicion}

\begin{Proposicion}[medias de Cesàro también convergen]\label{medias-convergen}
  Si \(S_N f \xrightarrow{L^2} f\), entonces \(\sigma_N f \xrightarrow{L^2} f\).  
\end{Proposicion}
\begin{proof}
  Sea \(\epsilon > 0\). Notemos que 
  \begin{align*}
    \normL{2}{f - \sigma_N f}
    &=
    \frac{1}{N} \normL{2}{ Nf - \sum_{0\le n\le N-1} S_n f}
    \\&\le
    \frac{1}{N} \sum_{n=0}^{N-1} \normL{2}{f - S_n f}
  \end{align*}
  Sea \(N_0\) tal que \(\normL{2}{f - S_{n} f} \le \epsilon\) para \(n\ge N_0\). 
  Tomando \(N \gg N_0\),
  \begin{align*}
    \normL{2}{f - \sigma_N f}
    &=
    \frac{1}{N} \underbrace{\sum_{n=0}^{N_0} \normL{2}{f - S_n f}}_{<\infty}
    +
    \frac{1}{N} \sum_{n=N_0+1}^{N} \normL{2}{f - S_n f}
    \\&\le
    \frac{1}{N} C
    +
    \epsilon C'
  \end{align*}
  Tomando \(N\) lo suficientemente grande podemos hacer que \(1/N < \epsilon\) de lo que se 
  sigue el resultado.
\end{proof}

\begin{Teorema}[Fejèr]
  Si \(\sigma_N f \to f\) en \(L^2\) y \(f\in C(\T)\), entonces
  \begin{displaymath}
    \sigma_N f \xrightarrow{\text{unif}} f 
    \quad\text{en}\quad\T.
  \end{displaymath}
\end{Teorema}

Si vale el Teorema de Fejèr, entonces vale el Teorema~\ref{main:thm} para funciones
continuas. En efecto,
usando la Proposición~\ref{medias-convergen} tenemos que
\begin{align*}
  \hat f(n) = 0 \quad \forall N\in\Z
  &\Rightarrow
  S_N f \equiv 0 \quad \forall N\in \Z
  \\&\Rightarrow
  \sigma_N f \equiv 0 \quad \forall N\in\Z \quad \text{(por la proposición)}
  \\&\Rightarrow
  \lim_{N\to\infty} \sigma_N f = f \quad\text{(por Fejér)}
  \\&\Rightarrow
  f = 0 \,\,c.t.p.
\end{align*}
Así que la nueva tarea es demostrar el teorema de Fejèr.

\begin{Proposicion}[Caracterización de la media de Cesàro y el kernel de Fejèr]
  Si \(f\in L^2\), entonces
  \begin{displaymath}
    \sigma_N f(x) = \int_{\T} F_N(x-t) f(t) \, dt
  \end{displaymath}
  donde
  \begin{displaymath}
    F_N(x) = 
    \begin{cases}
      \frac{N}{2\pi} &, x=0\\
      \frac{1}{2\pi N} \frac{\sin^2(Nx/2)}{\sin^2(x/2)} &, x\ne0
    \end{cases}
  \end{displaymath}
  es el kernel de Fejèr.
\end{Proposicion}
\begin{proof}
  \begin{align*}
    N \sigma_{N} f (x)
    =
    \int_{\T} f(t) \left( D_0(x-t) + \cdots + D_{N-1}(x-t) ) \, dt
  \end{align*}
  Luego,
  \begin{align*}
    \sum_{k=0}^{N-1} D_k (x)
    &=
    \frac{1}{2\pi} 
    \sum_{k=0}^{N-1} \frac{\sin((k+1/2)x)}{\sin(x/2)}
    \\&=
    \frac{1}{2\pi \sin(x/2)} 
    \sum_{k=0}^{N-1} \sin((k+1/2)x)
    \\&=
    \frac{1}{4\pi \sin^2(x/2)} 
    \sum_{k=0}^{N-1} 2\sin\left( (k+\frac{1}{2}) x \right) \sin(x/2)
    \\&=
    \frac{1}{4\pi \sin^2(x/2)} 
    \sum_{k=0}^{N-1} 
    \left(
      \cos\left( (k+\frac{1}{2})x - x/2 \right) 
      - 
      \cos\left( (k+\frac{1}{2})x + x/2 \right)
    \right)
    \\&=
    \frac{1}{4\pi \sin^2(x/2)} 
    \sum_{k=0}^{N-1} 
    \left(
      \cos\left( kx \right)
      - 
      \cos\left( (k+1)x \right) 
    \right)
    \\&=
    \frac{1 - \cos(Nx)}{2} \frac{1}{2\pi \sin^2(x/2)} 
    \\&=
    \frac{\sin^2(Nx/2)}{2\pi\sin^2(x/2)}.
  \end{align*}
\end{proof}

\begin{Proposicion}[Propiedades del kernel de Fejèr]
  \(F_N(x)\) es par, \(2\pi\){-}periódica, suave, integra \(1\) en \(\T\) y es positiva.
  Además, para \(\delta \le \abs{x} \le \pi\) se cumple que 
  \begin{displaymath}
    \abs{F_N(x)} \le \frac{1}{2\pi N \sin^{2}(\delta/x)}.
  \end{displaymath}
\end{Proposicion}

\begin{Definicion}[Buenos Kernels]
  Una familia \(\lbrace K_n \rbrace_{n\in \N}\) se dice de buenos kernels en \(L^1\) si:
  \begin{enumerate}
    \item \(\int_{\T} K_n(x) \,dx = 1\);
    \item \(\sup_{n\in\N} \int_{\T} \abs{K_n(x)}\, dx < \infty\), y
    \item \(\int_{\delta \le \abs{x} \le \pi} \abs{K_n(x)} \, dx \xrightarrow{n\to\infty} 0\) 
    para todo \(\delta > 0\).
  \end{enumerate}
\end{Definicion}

Por lo visto anteriormente, la familia de los kernels de Fejèr son buenos kernels.

\begin{Teorema}[Convolución y convergencia con buenos kernels]
  Si \(\lbrace K_n \rbrace_n\) es una familia de buenos kernels en \(L^1\) y 
  \(f\in C(\T)\), entonces
  \begin{displaymath}
    K_N\ast f = f \ast K_N \xrightarrow{\text{unif}} f
    \quad\text{ en } \T. 
  \end{displaymath}
\end{Teorema}
\begin{proof}
  Hay que acotar uniformemente la diferencia
  \begin{displaymath}
    \abs{ K_N\ast f(x) - f(x)}
  \end{displaymath}
  Recordando que \(\int_{\T} K_N(y) \, dy = 1\) tenemos que
  \begin{displaymath}
    \abs{ K_N \ast f(x) - f(x)}
    \le
    \int_{\T} \abs{f(x) - f(x-y)} K_N(y)\, dy
    \eqqcolon I
  \end{displaymath}
  Sea \(\epsilon > 0\).
  Como la función \(f\) es continua, existe \(\delta\) tal que \(\abs{f(x) - f(x-y)} \le \epsilon\)
  si \(\abs{y} \le \delta\). Luego,
  \begin{align*}
    I 
    &\le 
    \int_{\abs{y} \le \delta}
      \abs{ f(x) - f(x-y) } K_N(y)\, dy
    +
    \int_{\delta \le \abs{y} \le \pi}
      \abs{ f(x) - f(x-y) } K_N(y)\, dy
    \\&\le
    \epsilon M 
    + 
    C
    \int_{\delta \le \abs{y} \le \pi} \abs{K_N(y)}\, dy 
  \end{align*}
  donde \(M\) sale de que \(\lbrace K_N\rbrace_{N}\) es una familia de buenos kernels
  (propiedad (2)) y \(C \ge 2\sup_{\T} \abs{f}\). Por último, usando la propiedad
  (3) de los buenos kernels se tiene que para \(N\) suficientemente grande
  \begin{displaymath}
    \int_{\delta \le \abs{y} \le \pi} \abs{K_N(y)}\, dy 
    \le \epsilon.
  \end{displaymath}
  De esta forma se tiene que \(\abs{ f(x) - K_N \ast f(x) } \le \epsilon (M + C)\) 
  independiente de \(x\) y por la arbitrariedad de \(\epsilon\) se concluye la convergencia. 
\end{proof}

\begin{Corolario}
  Si \(f\in C(\T)\), entonces \(\sigma_N f \xrightarrow{\text{unif}} f\) cuando \(N\to\infty\).   
\end{Corolario}
\begin{proof}
  \(\sigma_N f = F_N \ast f\) y \(F_N\) es una familia de buenos kernels.  
\end{proof}

\begin{Corolario}\label{cor:coeff f nulo impl f nulo}
  Si \(f\in C(\T)\) y \(\hat f(n) = 0\) para todo \(n\in \Z\), entonces \(f = 0\).   
\end{Corolario}
\begin{proof}
  Si \(\hat f(n) = 0\), entonces \(S_N f = 0\) y por lo tanto \(\sigma_N f = 0\). Por
  el corolario anterior, \(f = 0\). 
\end{proof}

\begin{Corolario}
  Si \(f\in C(\T)\) es tal que su serie de Fourier converge absoluta y uniformemente, .i.e.
  \begin{displaymath}
    \sum_{n\in\Z} \abs{ \hat f(n) e_n(x) } < \infty.
  \end{displaymath}
  Entonces \(S_N f \xrightarrow{\text{unif}} f\).
\end{Corolario}
\begin{proof}
\end{proof}

Esto nos da convergencia para funciones continuas. La idea ahora es aproximar las funciones
en \(L^2\).

\begin{Proposicion}
  Si \(f\in L^2\), entonces \(\normL{2}{\sigma_N f} \le \normL{2}{f}\).
\end{Proposicion}
\begin{proof}
  Usando la definición y la desigualdad de Bessel
  \begin{displaymath}
    \normL{2}{\sigma_N f}
    \le
    \frac{1}{N} \sum_{n=0}^{N-1} \normL{2}{S_n f}
    \le
    \frac{1}{N} \sum_{n=0}^{N-1} \normL{2}{f}
    \le \normL{2}{f}.
  \end{displaymath}
\end{proof}

El resultado vale para \(1 \le p <\infty\). 

\begin{Teorema}
  Sea \(f\in L^p(\T)\) (\(1\le p<\infty)\), entonces \(\sigma_N f \to f\) en \(L^p\).    
\end{Teorema}
\begin{proof}
  Sea \(\epsilon > 0\).
  Como \(C\) es denso en \(L^p\), existe \(g\in C\) tal que
  \begin{displaymath}
    \normL{p}{ f - g } \le \epsilon.
  \end{displaymath}
  Luego,
  \begin{displaymath}
    \normL{p}{ f-\sigma_N f }
    \le
    \normL{p}{ f - g }
    +
    \normL{p}{ g - \sigma_N g }
    +
    \normL{p}{\sigma_N g - \sigma_N f}
  \end{displaymath}
  Por~ existe \(N_0\) tal que \(\normL{p}{ g - \sigma_N g } \le \epsilon\).  
  Por otro lado,
  \begin{align*}
    \normL{p}{\sigma_N g - \sigma_N f}
    &\le
    \normL{p}{\sigma_N (g-f)}
    \\&\le
    \normL{p}{g-f}.
  \end{align*}
  Así, para \(N>N_0\) suficientemente grande
  \begin{displaymath}
    \normL{p}{f-\sigma_N} \le 3\epsilon.
  \end{displaymath}
\end{proof}

\begin{Corolario}
  Si \(f\in L^2\), \(S_N f \xrightarrow{L^2} f\).
\end{Corolario}
\begin{proof}
\end{proof}

\begin{Lema}[Riemann{-}Lebesgue]
  Si \(f\in L^1\), entonces \(\hat f(n) \xrightarrow{\abs{n}\to\infty} 0\).  
\end{Lema}
\begin{proof}
  Notar que si \(\abs{n}>N\) 
  \begin{displaymath}
    \widehat{\sigma_N f}(n)
    =
    \frac{1}{N}
    \sum_{k=0}^{N-1}
      \sum_{\abs{j} \le k}
        \hat{f}(j)
        \int_{T} e_j(t) e_n(t)\, dt
    =
    0
  \end{displaymath}
  Sea \(\epsilon > 0\). Luego, tomando \(N\) suficientemente grande se tiene que 
  \(\normL{1}{\sigma_N f - f} < \epsilon\). Por lo anterior se sigue que
  \begin{displaymath}
    \abs{ \hat f(n) }
    =
    \abs{ \widehat{\sigma_N f}(n) - \hat f(n) }
    \le
    \normL{1}{\sigma_N f - f} < \epsilon.
  \end{displaymath}
\end{proof}

\begin{Definicion}[Operador de Fourier]
  Para \(f\in L^1\), el operador de Fourier se define como:
  \begin{displaymath}
    \hat f \coloneqq \F f \coloneqq (\hat f(n))_{n\in \Z}. 
  \end{displaymath}
\end{Definicion}

\begin{Proposicion}[Inyección en \(\hat{c_0}\)]
  Si \(f\in L^1\), entonces \(L^1 \xrightarrow{\F} \hat{c}_0\) es una aplicación lineal, acotada
  e inyectiva.
  \begin{displaymath}
    \hat c_0 \coloneqq \lbrace (a_n)_{n\in\Z}\colon \lim_{\abs{n} \to \infty} a_n = 0 \rbrace.
  \end{displaymath}
\end{Proposicion}
\begin{proof}
  Por Riemann{-}Lebesgue la aplicación está bien definida. Veamos el resto de propiedades:
  
  \underline{lineal}: se sigue por la linealidad de la integral.

  \underline{acotada}: se sigue porque \(f\in L^1\). 

  \underline{inyectiva}: se sigue de aplicar el Corolario~\ref{cor:coeff f nulo impl f nulo} y aproximando
  por funciones continuas.
\end{proof}

Veamos porqué la aplicación \(\F \colon L^1 \to \hat c_0\) no es sobreyectiva. Supongamos, buscando
una contradicción, que lo fuese. Por el resto de las propiedades, \(\F\) es un isomorfismo continuo
y en particular existe \(\F^{-1}\) su inversa continua. Luego,
\begin{displaymath}
  \normL{1}{f} = \normL{1}{\F^{-1} \F f} \le C \norm{\F f}_{\ell^{\infty}}. 
\end{displaymath}
para cualquier \(f\in L^1\). Consideremos \(f(x) = D_N(x)\). Entonces
\begin{displaymath}
  \hat f(x) = \begin{cases}
    \frac{1}{\sqrt{2\pi}} &, \abs{n} \le N\\
    0 &, \abs{n} > N\\
  \end{cases}
\end{displaymath}
así que \(\norm{\F f}_{\ell^{\infty}} = 1/\sqrt{2\pi}\). Sin embargo,

\begin{Proposicion}
  \begin{displaymath}
    \normL{1}{D_N} \ge c\log(N).
  \end{displaymath}
\end{Proposicion}
\begin{proof}
\begin{alignat*}{2}
  \normL{1}{D_N}
  &=
  \int_{-\pi}^{\pi} 
    \abs{D_N(x)} 
  \,dx
  &&=
  \frac{1}{\pi}
  \int_{0}^{\pi} 
    \frac{\abs{\sin((N+1/2) x)}}{ \sin(x/2) }
  \,dx
  \\&\ge
  \frac{2}{\pi}
  \int_{0}^{\pi} 
    \frac{\abs{\sin((N+1/2) x)}}{ x }
  \,dx
  \\&=
  \frac{2}{\pi}
  \int_{0}^{\pi (N+1/2)} 
    \frac{\abs{\sin(u)}}{ u }
  \, du
  &&\ge
  \frac{2}{\pi}
  \int_{0}^{\pi N} 
    \frac{\abs{\sin(u)}}{ u }
  \, du
  \\&\ge
  \frac{2}{\pi}
  \sum_{n=1}^{N}
    \int_{(n-1)\pi}^{n\pi} 
      \frac{\abs{\sin(u)}}{ u }
    \, du
  &&\ge
  \frac{2}{\pi}
  \sum_{n=1}^{N}
    \frac{1}{n\pi}
    \underbrace{
    \int_{(n-1)\pi}^{n\pi} 
      \abs{\sin(u)}
    \, du}_{C_1}
  \\&\ge
  \frac{2 C_1}{\pi^2}
  \sum_{n=1}^{N}
    \frac{1}{n}
  &&\ge C \log(N).
\end{alignat*}
\end{proof}

Vimos que para funciones \(f\in L^2\) se tiene que \(S_N f \to f\) en \(L^2\). De hecho,
\(S_N f \to f\) puntualmente c.t.p. (Teorema de Carlerson). 
En lo que sigue vamos a probar que la divergencia puntual pasa en conjuntos grandes.

\begin{Teorema}[Banach-Steinhaus]
  Sea \(X\) es un espacio Banach y \(Y\) un espacio normado. 
  Sean \(\lbrace T_n \rbrace_{n\in\N} \subset B(X,Y)\) una familia de operadores
  lineales acotados de \(X\) en \(Y\). Entonces las siguientes son mutuamente excluyentes:
  \begin{enumerate}
    \item \(\sup_{n\in\N} \norm{T_n} < \infty\)
    \item existe \(A\) un conjunto genérico de tipo \(G_{\delta}\) (intersección numerable de abiertos) 
    en \(X\) tal que
    \begin{displaymath}
      \sup_{n} \norm{T_n x} = \infty
      \quad\forall x\in X.
    \end{displaymath}
  \end{enumerate}
\end{Teorema}
\begin{proof}
Definamos \(\psi(x) \coloneqq \sup_{n\in \N} \norm{T_n x}\) y consideremos el conjunto
abierto 
\begin{displaymath}
  U_n 
  = \left\{ x\in X \colon \psi(x) > n \right\}
  = \bigcup_{k\in\N} \left\{ x\in X \colon \norm{T_k x} > n \right\} 
  = \bigcup_{k\in\N} T_k^{-1} \left( Y\setminus B(0,n) \right).
\end{displaymath}

Si la familia de conjuntos \(U_n\) es densa en \(X\), entonces el conjunto
\begin{displaymath}
  A \coloneqq \bigcap_{n\ge 1} U_n 
\end{displaymath}
es \(G_{\delta}\) y satisface que 
\begin{displaymath}
  \sup_{k\in\N} \norm{T_k x} = \infty
  \quad\forall x\in A.
\end{displaymath}
por lo tanto la condición (1) no es posible. 

Por otro lado, si al menos uno no es denso, dígase, \(U_n\), entonces el complemento
\(X\setminus U_n\) contiene una bola \(B(u,r)\) para algún \(r>0\) y \(u\in X\setminus U_n\).
Luego, para todo \(x\in B(u,r)\) se tiene que
\begin{displaymath}
  n \ge \psi(x) \ge \norm{T_k x} \quad \forall k\in\N.
\end{displaymath}
Haciendo el cambio de coordenadas \(x = u+y\) con \(y\in B(0,r)\) tenemos que  
\begin{align*}
  \norm{T_k (u+y)} &\le n
  \\\Rightarrow
  \norm{T_k y}
  &\le
  \norm{T_k u} + \norm{T_k (u-y)}
  \le 2n
  \\\Rightarrow
  \norm{T_k y}
  &\le
  \frac{2n}{r}
  \norm{y}
  \quad\forall y\in B(0,1).
\end{align*}
Se sigue que \(\norm{T_k} < \frac{2n}{r}\) para todo \(k\in \N\). En particular, 
\(\norm{T_k x} < \infty\) para todo \(x\in X\).
\end{proof}

\begin{Corolario}
  Sea \(X\) un espacio de Banach y \(Y\) un espacio normado. Sean \(\lbrace T_k \rbrace_{k}\)
  operadores lineales acotados de \(X\) a \(Y\). Supongamos que
  \begin{displaymath}
    \forall x\in X \colon\quad Tx \coloneqq \lim_{k\to\infty} T_k x \in Y
  \end{displaymath}
  entonces \(T\in B(X,Y)\) y \(\norm{T} \le \liminf_{k\to\infty} \norm{T_k} < \infty\).  
\end{Corolario}
\begin{proof}
  Como los operadores son acotados y el límite puntual existe, por Banach{-}Steinhaus 
  \begin{displaymath}
    \sup_{k} \norm{T_k} < \infty.
  \end{displaymath}
  Es sencillo verificar que \(T\) es lineal. 
  Por otro lado,
  \begin{displaymath}
    \norm{Tx}
    =
    \lim_{k\to\infty} \norm{T_k x}
    \le
    \sup_{n\in \N} \inf_{k\ge n} \norm{T_k x}
    \le
    \sup_{n\in \N} \inf_{k\ge n} \norm{T_k}\norm{x}
  \end{displaymath}
  Diviendo por \(\norm{x}\) tenemos lo pedido.  
\end{proof}

\begin{Teorema}[Divergencia Puntual]\label{teo:div-puntual}
  Para todo \(x\in\T\), existe un conjunto genérico \(A_x \in C(\T)\) tal que
  \(\sup_{N} \abs{S_N f(x)} = \infty\) para todo \(f\in A_x\).  
\end{Teorema}
\begin{proof}
  Basta demostrar para el caso \(x=0\) pues...
  Consideremos \(S_N^0 \colon C(\T) \to \C\) la evaluación de la suma parcial de fourier
  en \(x=0\) .i.e.
  \begin{displaymath}
    S_N^0 f = S_N f(0).
  \end{displaymath}
  
  Notar que \(S_N^0 f = \int_{\T} D_N(y) f(y) \, dy\), así que
  \begin{displaymath}
    \norm{S_N^0} \le \norm{D_N}_{L^1}.
  \end{displaymath}
  Notar que la igualdad se alcanza en \(f(y) = sgn(D_N(y))\) y en tal caso
  \begin{displaymath}
    \norm{S_N^0} = \norm{D_N}_{L^1} \ge C\log N \xrightarrow{N\to\infty} \infty.
  \end{displaymath}
  El problema es que \(f(y)\) no es continua, no obstante, \(f(y) \in L^1\) y podemos
  aproximar por funciones continuas. Así, por Banach-Steinhaus, existe un conjunto genérico
  en \(C(\T)\) donde el operador \(S_N^0\) diverge.  
\end{proof}

\noindent\tikz{\draw (0,0) -- (\textwidth,0) node[midway,above,fill=white]{Problemas}; }

\begin{Problema}
  Sea \(f\in L^2\) una función Lipschitz continua en \(x_0 \in [-\pi, \pi]\). 
  Demuestre que su serie de Fourier \(S_N f\) converge puntualmente a \(f\)
  en \(x = x_0\). 
\end{Problema}
\begin{proof}
  Sea \(\epsilon > 0\). Queremos probar que para \(N\) suficientemente grande se tiene que
  \begin{displaymath}
    E = \abs{ f(x_0) - S_N f(x_0) } < \epsilon.
  \end{displaymath}

  Usando la caracterización con el kernel de Dirichlet tenemos que
  \begin{displaymath}
    E = \abs{ \int_{\T} (f(x_0) - f(x_0 - t)) D_N(t) \, dt }.
  \end{displaymath}
  Dado que \(f\) es Lipschitz continua en \(x_0\), existe \(\delta > 0\) tal que
  \begin{displaymath}
    \abs{t} < \delta \implies \abs{ f(x_0) - f(x_0 - t) } < \epsilon.
  \end{displaymath}
  Por otro lado, por las propiedades del kernel de Dirichlet, para \(N\) suficientemente grande
  se cumple que
  \begin{displaymath}
    \int_{\abs{t} \ge \delta} \abs{ D_N(t) } \, dt < \epsilon.
  \end{displaymath}
  Así, separando la integral tenemos que
  \begin{displaymath}
    E \le 
    \epsilon \int_{\abs{t}\le \delta} \abs{t D_N(t)} \, dt
    +
    \epsilon 2 \esssup_{\T} \abs{f}
    \le 
    C \epsilon.
  \end{displaymath}
\end{proof}

\begin{Problema}
  Usando la serie de Fourier de \(\abs{x}\) en \([-\pi, \pi]\), demuestre que
  \begin{displaymath}
    \sum_{n=1}^{\infty} \frac{1}{n^2}
    =
    \frac{\pi^2}{6}.
  \end{displaymath}
\end{Problema}
\begin{proof}
  Calculemos los coeficientes. Para \(n\ne 0\) 
  \begin{alignat*}{1}
    2\pi \hat f(n) 
    &= \int_{-\pi}^{\pi} \abs{x} e^{-inx} \, dx
    \\&= \int_{-\pi}^{0} -x e^{-inx} \, dx
      + \int_{0}^{\pi} x e^{-inx} \, dx
    \\&= \int_{0}^{\pi} x (e^{-inx} + e^{inx}) \, dx
    \\&= 2 \int_{0}^{\pi} x\cos(nx) \, dx
    \\&= 2 \left(
      \left[ \frac{x \sin(nx)}{n} \right]_0^{\pi}
      -
      \frac{1}{n} \int_{0}^{\pi} \sin(nx)\, dx
    \right) 
    \\&=
    2 \frac{(-1)^{n} - 1}{n^2}
  \end{alignat*}
  Mientras que para \(n=0\) nos da \(\hat f(0) = \pi/2\). De esta forma, dado que \(\abs{x}\) es
  Lipschitz en \(0\), se tiene que 
  \begin{align*}
    \abs{0} 
    &= \frac{\pi}{2} + \frac{2}{\pi} \sum_{n=1}^{\infty} \frac{(-1)^n - 1}{n^2}.
    \\&= \frac{\pi}{2} - \frac{4}{\pi} \sum_{n=1}^{\infty} \frac{1}{(2n-1)^2}.
  \end{align*}
  Despejando tenemos 
  \begin{displaymath}
    \sum_{n=1}^{\infty} \frac{1}{(2n-1)^2}
    =
    \frac{\pi^2}{8}.
  \end{displaymath}

  Por otro lado,
  \begin{displaymath}
    \sum_{n=1}^{\infty} 
      \frac{1}{n^2}
    =
    \sum_{n=1}^{\infty} 
      \frac{1}{(2n-1)^2}
    +
    \sum_{n=1}^{\infty} 
      \frac{1}{(2n)^2}
    =
    \frac{\pi^2}{8}
    +
    \frac{1}{4}
    \sum_{n=1}^{\infty} 
      \frac{1}{n^2}
  \end{displaymath}
  y despejando se obtiene el resultado.
\end{proof}

\begin{Problema}
  Sea \(\alpha \not\in \Z\). Demuestre que la serie de Fourier en \([0,2\pi]\) de
  \begin{displaymath}
    \frac{\pi}{\sin(\pi\alpha)} e^{i(\pi-x)\alpha}
  \end{displaymath}
  está dada por
  \begin{displaymath}
    \sum_{n\in\Z} \frac{e^{inx}}{n+\alpha}.
  \end{displaymath}
  Concluya, usando la identidad de Parseval, que
  \begin{displaymath}
    \sum_{n\in\Z} \frac{1}{(n+\alpha)^2}
    =
    \frac{\pi^2}{(\sin(\pi\alpha))^2}.
  \end{displaymath}
\end{Problema}
\begin{proof}
  Denotemos a la función por \(f(x)\) y computemos los coeficientes.
  \begin{alignat*}{1}
    2\pi \hat f(n)
    &= 
    \frac{ e^{i\pi\alpha}\pi }{ \sin(\pi\alpha) }
    \int_{0}^{2\pi} 
      e^{-ix(\alpha+n)} \, dx
    \\&= 
    - \frac{ e^{i\pi\alpha}\pi }{ i(\alpha+n) \sin(\pi\alpha) } 
    ( e^{-i2\pi(\alpha+n)} - 1 ) 
    \\&= 
    \frac{\pi e^{i\pi\alpha} (e^{-i2\pi(\alpha+n)}-1)}{i(\alpha+n) \sin(\pi\alpha)}
    \\&= 
    \frac{2\pi \sin(\pi\alpha)}{(\alpha+n) \sin(\pi\alpha)}.
  \end{alignat*}
  Así, la serie de Fourier es
  \begin{displaymath}
    \sum_{n\in\Z} \frac{e^{inx}}{\alpha+n}.
  \end{displaymath}

  Aplicando Parseval tenemos que
  \begin{displaymath}
    \sum_{n\in\Z} \frac{1}{(\alpha+n)^2}
    =
    \normL{2}{f}^2
    =
    \frac{1}{2\pi}
    \frac{\pi^2}{\sin(\pi\alpha)^2}
    \int_{0}^{2\pi} 1\, dx
    =
    \frac{\pi^2}{\sin(\pi\alpha)^2}
  \end{displaymath}
\end{proof}

\begin{Problema}
\begin{enumerate}
  \item Demuestre que
  \begin{displaymath}
    \int_{0}^{\infty} \frac{\sin(x)}{x} \, dx = \frac{\pi}{2}.
  \end{displaymath}
  \textit{Sugerencia:} use que \(\int_{-\pi}^{\pi} D_N(x) \, dx = 1\) y note que
  \((1/\sin(x/2)) - (2/x)\) es continuo en \([-\pi,\pi]\). Use el lema de Riemann-Lebesgue.  
\end{enumerate}
\end{Problema}

\begin{Problema}
  Sea \(a = (a_n)_{n\in\N}\) una sucesión tal que para todo \(x\in c_0\) se tiene que
  \(\sum_{n\ge 1} \abs{a_n x_n} \) converge. Demuestre que \(a\in \ell^1\).  
\end{Problema}
\begin{proof}
  Debemos probar que \(\sum_{n\ge 1} \abs{a_n} < \infty\).
  Consideremos el operador
  \begin{align*}
    S_N \colon c_0 &\to \ell^{\infty} 
    \\ x&\mapsto \sum_{n=1}^{N} a_n x_n.
  \end{align*}
  Como son sumas finitas es lineal y acotado. Además, por la propiedad de \(a\) está bien definido. 
  Por otro lado, \(\sup_{N\in \N} S_N x < \infty\) pues es la propiedad de \(a\). Luego, por 
  Banach-Steinhaus, \(\sup_{N\in \N} \norm{ S_N x } < \infty\). Dado que
  \begin{displaymath}
    S_N (sng(a_n))_{n\le N} = \sum_{n\le N} \abs{a_n}
  \end{displaymath}
  se concluye lo pedido.
\end{proof}

\begin{Problema}
  Sean \(V,W\) espacios de Banach y \(a\colon V\times W \to \K\) bilineal tal que
  \begin{itemize}
    \item para todo \(v\in V\), la función \(w\to a(v,w)\) es continua en \(W\)   
    \item para todo \(w\in W\), la función \(v\to a(v,w)\) es continua en \(V\)   
  \end{itemize}
  Demuestre que existe \(C\ge 0\) tal que para todo \(v\in V\) y \(w\in W\) se tiene que
  \begin{displaymath}
    \abs{a(v,w)} \le C \norm{v} \norm{w}.
  \end{displaymath}
  Concluya que \(a\) es continua. 
\end{Problema}
\begin{proof}
\begin{displaymath}
  \norm{a(v,w)} \le \norm{a(,w)} \norm{w} \le \norm{a} \norm{v} \norm{w}.
\end{displaymath}
Directo de la cota.
\end{proof}
\end{document}
